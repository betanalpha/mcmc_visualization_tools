% Options for packages loaded elsewhere
\PassOptionsToPackage{unicode}{hyperref}
\PassOptionsToPackage{hyphens}{url}
\PassOptionsToPackage{dvipsnames,svgnames,x11names}{xcolor}
%
\documentclass[
  letterpaper,
  DIV=11,
  numbers=noendperiod]{scrartcl}

\usepackage{amsmath,amssymb}
\usepackage{iftex}
\ifPDFTeX
  \usepackage[T1]{fontenc}
  \usepackage[utf8]{inputenc}
  \usepackage{textcomp} % provide euro and other symbols
\else % if luatex or xetex
  \usepackage{unicode-math}
  \defaultfontfeatures{Scale=MatchLowercase}
  \defaultfontfeatures[\rmfamily]{Ligatures=TeX,Scale=1}
\fi
\usepackage{lmodern}
\ifPDFTeX\else  
    % xetex/luatex font selection
\fi
% Use upquote if available, for straight quotes in verbatim environments
\IfFileExists{upquote.sty}{\usepackage{upquote}}{}
\IfFileExists{microtype.sty}{% use microtype if available
  \usepackage[]{microtype}
  \UseMicrotypeSet[protrusion]{basicmath} % disable protrusion for tt fonts
}{}
\makeatletter
\@ifundefined{KOMAClassName}{% if non-KOMA class
  \IfFileExists{parskip.sty}{%
    \usepackage{parskip}
  }{% else
    \setlength{\parindent}{0pt}
    \setlength{\parskip}{6pt plus 2pt minus 1pt}}
}{% if KOMA class
  \KOMAoptions{parskip=half}}
\makeatother
\usepackage{xcolor}
\setlength{\emergencystretch}{3em} % prevent overfull lines
\setcounter{secnumdepth}{5}
% Make \paragraph and \subparagraph free-standing
\ifx\paragraph\undefined\else
  \let\oldparagraph\paragraph
  \renewcommand{\paragraph}[1]{\oldparagraph{#1}\mbox{}}
\fi
\ifx\subparagraph\undefined\else
  \let\oldsubparagraph\subparagraph
  \renewcommand{\subparagraph}[1]{\oldsubparagraph{#1}\mbox{}}
\fi

\usepackage{color}
\usepackage{fancyvrb}
\newcommand{\VerbBar}{|}
\newcommand{\VERB}{\Verb[commandchars=\\\{\}]}
\DefineVerbatimEnvironment{Highlighting}{Verbatim}{commandchars=\\\{\}}
% Add ',fontsize=\small' for more characters per line
\usepackage{framed}
\definecolor{shadecolor}{RGB}{241,243,245}
\newenvironment{Shaded}{\begin{snugshade}}{\end{snugshade}}
\newcommand{\AlertTok}[1]{\textcolor[rgb]{0.68,0.00,0.00}{#1}}
\newcommand{\AnnotationTok}[1]{\textcolor[rgb]{0.37,0.37,0.37}{#1}}
\newcommand{\AttributeTok}[1]{\textcolor[rgb]{0.40,0.45,0.13}{#1}}
\newcommand{\BaseNTok}[1]{\textcolor[rgb]{0.68,0.00,0.00}{#1}}
\newcommand{\BuiltInTok}[1]{\textcolor[rgb]{0.00,0.23,0.31}{#1}}
\newcommand{\CharTok}[1]{\textcolor[rgb]{0.13,0.47,0.30}{#1}}
\newcommand{\CommentTok}[1]{\textcolor[rgb]{0.37,0.37,0.37}{#1}}
\newcommand{\CommentVarTok}[1]{\textcolor[rgb]{0.37,0.37,0.37}{\textit{#1}}}
\newcommand{\ConstantTok}[1]{\textcolor[rgb]{0.56,0.35,0.01}{#1}}
\newcommand{\ControlFlowTok}[1]{\textcolor[rgb]{0.00,0.23,0.31}{#1}}
\newcommand{\DataTypeTok}[1]{\textcolor[rgb]{0.68,0.00,0.00}{#1}}
\newcommand{\DecValTok}[1]{\textcolor[rgb]{0.68,0.00,0.00}{#1}}
\newcommand{\DocumentationTok}[1]{\textcolor[rgb]{0.37,0.37,0.37}{\textit{#1}}}
\newcommand{\ErrorTok}[1]{\textcolor[rgb]{0.68,0.00,0.00}{#1}}
\newcommand{\ExtensionTok}[1]{\textcolor[rgb]{0.00,0.23,0.31}{#1}}
\newcommand{\FloatTok}[1]{\textcolor[rgb]{0.68,0.00,0.00}{#1}}
\newcommand{\FunctionTok}[1]{\textcolor[rgb]{0.28,0.35,0.67}{#1}}
\newcommand{\ImportTok}[1]{\textcolor[rgb]{0.00,0.46,0.62}{#1}}
\newcommand{\InformationTok}[1]{\textcolor[rgb]{0.37,0.37,0.37}{#1}}
\newcommand{\KeywordTok}[1]{\textcolor[rgb]{0.00,0.23,0.31}{#1}}
\newcommand{\NormalTok}[1]{\textcolor[rgb]{0.00,0.23,0.31}{#1}}
\newcommand{\OperatorTok}[1]{\textcolor[rgb]{0.37,0.37,0.37}{#1}}
\newcommand{\OtherTok}[1]{\textcolor[rgb]{0.00,0.23,0.31}{#1}}
\newcommand{\PreprocessorTok}[1]{\textcolor[rgb]{0.68,0.00,0.00}{#1}}
\newcommand{\RegionMarkerTok}[1]{\textcolor[rgb]{0.00,0.23,0.31}{#1}}
\newcommand{\SpecialCharTok}[1]{\textcolor[rgb]{0.37,0.37,0.37}{#1}}
\newcommand{\SpecialStringTok}[1]{\textcolor[rgb]{0.13,0.47,0.30}{#1}}
\newcommand{\StringTok}[1]{\textcolor[rgb]{0.13,0.47,0.30}{#1}}
\newcommand{\VariableTok}[1]{\textcolor[rgb]{0.07,0.07,0.07}{#1}}
\newcommand{\VerbatimStringTok}[1]{\textcolor[rgb]{0.13,0.47,0.30}{#1}}
\newcommand{\WarningTok}[1]{\textcolor[rgb]{0.37,0.37,0.37}{\textit{#1}}}

\providecommand{\tightlist}{%
  \setlength{\itemsep}{0pt}\setlength{\parskip}{0pt}}\usepackage{longtable,booktabs,array}
\usepackage{calc} % for calculating minipage widths
% Correct order of tables after \paragraph or \subparagraph
\usepackage{etoolbox}
\makeatletter
\patchcmd\longtable{\par}{\if@noskipsec\mbox{}\fi\par}{}{}
\makeatother
% Allow footnotes in longtable head/foot
\IfFileExists{footnotehyper.sty}{\usepackage{footnotehyper}}{\usepackage{footnote}}
\makesavenoteenv{longtable}
\usepackage{graphicx}
\makeatletter
\def\maxwidth{\ifdim\Gin@nat@width>\linewidth\linewidth\else\Gin@nat@width\fi}
\def\maxheight{\ifdim\Gin@nat@height>\textheight\textheight\else\Gin@nat@height\fi}
\makeatother
% Scale images if necessary, so that they will not overflow the page
% margins by default, and it is still possible to overwrite the defaults
% using explicit options in \includegraphics[width, height, ...]{}
\setkeys{Gin}{width=\maxwidth,height=\maxheight,keepaspectratio}
% Set default figure placement to htbp
\makeatletter
\def\fps@figure{htbp}
\makeatother

\KOMAoption{captions}{tableheading}
\makeatletter
\@ifpackageloaded{caption}{}{\usepackage{caption}}
\AtBeginDocument{%
\ifdefined\contentsname
  \renewcommand*\contentsname{Table of contents}
\else
  \newcommand\contentsname{Table of contents}
\fi
\ifdefined\listfigurename
  \renewcommand*\listfigurename{List of Figures}
\else
  \newcommand\listfigurename{List of Figures}
\fi
\ifdefined\listtablename
  \renewcommand*\listtablename{List of Tables}
\else
  \newcommand\listtablename{List of Tables}
\fi
\ifdefined\figurename
  \renewcommand*\figurename{Figure}
\else
  \newcommand\figurename{Figure}
\fi
\ifdefined\tablename
  \renewcommand*\tablename{Table}
\else
  \newcommand\tablename{Table}
\fi
}
\@ifpackageloaded{float}{}{\usepackage{float}}
\floatstyle{ruled}
\@ifundefined{c@chapter}{\newfloat{codelisting}{h}{lop}}{\newfloat{codelisting}{h}{lop}[chapter]}
\floatname{codelisting}{Stan

Program}
\newcommand*\listoflistings{\listof{codelisting}{List of Listings}}
\makeatother
\makeatletter
\makeatother
\makeatletter
\@ifpackageloaded{caption}{}{\usepackage{caption}}
\@ifpackageloaded{subcaption}{}{\usepackage{subcaption}}
\makeatother
\ifLuaTeX
  \usepackage{selnolig}  % disable illegal ligatures
\fi
\usepackage{bookmark}

\IfFileExists{xurl.sty}{\usepackage{xurl}}{} % add URL line breaks if available
\urlstyle{same} % disable monospaced font for URLs
\hypersetup{
  pdftitle={Markov Chain Monte Carlo Visualization Functions},
  pdfauthor={Michael Betancourt},
  colorlinks=true,
  linkcolor={blue},
  filecolor={Maroon},
  citecolor={Blue},
  urlcolor={Blue},
  pdfcreator={LaTeX via pandoc}}

\title{Markov Chain Monte Carlo Visualization Functions}
\author{Michael Betancourt}
\date{2024-05-01}

\begin{document}
\maketitle

\renewcommand*\contentsname{Table of contents}
{
\hypersetup{linkcolor=}
\setcounter{tocdepth}{3}
\tableofcontents
}
In this note I will review a suite of \texttt{R} functions that
implement various visualizations of probabilistic behavior using the
output of a Markov chain Monte Carlo algorithm.

Most of these visualizations utilize nested quantile intervals to
visualize one-dimensional pushforward behavior as described in
\href{https://betanalpha.github.io/assets/chapters_html/transforming_probability_spaces.html\#sec:1d-pushforward-characterizations}{Chapter
7, Section 5} of my probability theory material. The individual
quantiles are consistently estimated as the empirical average of the
empirical quantiles derived from individual Markov chains. Because they
do not communicate the quantile estimator errors these visualizations
can be misleading if the Markov chains do not contain enough
information.

\section{Initial Setup}\label{initial-setup}

First and foremost we have to set up our local \texttt{R} environment.

\begin{Shaded}
\begin{Highlighting}[]
\FunctionTok{par}\NormalTok{(}\AttributeTok{family=}\StringTok{"serif"}\NormalTok{, }\AttributeTok{las=}\DecValTok{1}\NormalTok{, }\AttributeTok{bty=}\StringTok{"l"}\NormalTok{, }\AttributeTok{cex.axis=}\DecValTok{1}\NormalTok{, }\AttributeTok{cex.lab=}\DecValTok{1}\NormalTok{, }\AttributeTok{cex.main=}\DecValTok{1}\NormalTok{,}
    \AttributeTok{xaxs=}\StringTok{"i"}\NormalTok{, }\AttributeTok{yaxs=}\StringTok{"i"}\NormalTok{, }\AttributeTok{mar =} \FunctionTok{c}\NormalTok{(}\DecValTok{5}\NormalTok{, }\DecValTok{5}\NormalTok{, }\DecValTok{3}\NormalTok{, }\DecValTok{5}\NormalTok{))}
\end{Highlighting}
\end{Shaded}

This includes loading up \texttt{RStan}.

\begin{Shaded}
\begin{Highlighting}[]
\FunctionTok{library}\NormalTok{(rstan)}
\FunctionTok{rstan\_options}\NormalTok{(}\AttributeTok{auto\_write =} \ConstantTok{TRUE}\NormalTok{)            }\CommentTok{\# Cache compiled Stan programs}
\FunctionTok{options}\NormalTok{(}\AttributeTok{mc.cores =}\NormalTok{ parallel}\SpecialCharTok{::}\FunctionTok{detectCores}\NormalTok{()) }\CommentTok{\# Parallelize chains}
\NormalTok{parallel}\SpecialCharTok{:::}\FunctionTok{setDefaultClusterOptions}\NormalTok{(}\AttributeTok{setup\_strategy =} \StringTok{"sequential"}\NormalTok{)}
\end{Highlighting}
\end{Shaded}

Finally we'll load my recommended
\href{https://github.com/betanalpha/mcmc_diagnostics}{Markov chain Monte
Carlo analysis tools} and the visualization functions themselves.

\begin{Shaded}
\begin{Highlighting}[]
\NormalTok{util }\OtherTok{\textless{}{-}} \FunctionTok{new.env}\NormalTok{()}
\FunctionTok{source}\NormalTok{(}\StringTok{\textquotesingle{}mcmc\_analysis\_tools\_rstan.R\textquotesingle{}}\NormalTok{, }\AttributeTok{local=}\NormalTok{util)}
\end{Highlighting}
\end{Shaded}

\begin{Shaded}
\begin{Highlighting}[]
\FunctionTok{source}\NormalTok{(}\StringTok{\textquotesingle{}mcmc\_visualization\_tools.R\textquotesingle{}}\NormalTok{, }\AttributeTok{local=}\NormalTok{util)}
\end{Highlighting}
\end{Shaded}

\section{One-Dimensional Baseline
Function}\label{one-dimensional-baseline-function}

Our first example is one-dimensional curve-fitting, i.e.~regression,
model with a linear baseline function, \[
p(y_{n} \mid x_{n}, \alpha, \beta, \sigma)
=
\text{normal}(y_{n} \mid \alpha + \beta \, x_{n}, \sigma).
\]

\subsection{Data Exploration}\label{data-exploration}

What is data if not an opportunity to explore?

\begin{Shaded}
\begin{Highlighting}[]
\NormalTok{data }\OtherTok{\textless{}{-}} \FunctionTok{read\_rdump}\NormalTok{(}\StringTok{"data/uni\_data.R"}\NormalTok{)}
\end{Highlighting}
\end{Shaded}

The \texttt{plot\_line\_hist} function constructs a histogram but then
plots only its outline, without lines separating the interior histogram
bins. Here we can plot histograms summarizing the observed inputs \[
\{ \tilde{x}_{1}, \ldots, \tilde{x}_{n}, \ldots, \tilde{x}_{N} \}
\] and the observed outputs, \[
\{ \tilde{y}_{1}, \ldots, \tilde{y}_{n}, \ldots, \tilde{y}_{N} \}.
\]

\begin{Shaded}
\begin{Highlighting}[]
\FunctionTok{par}\NormalTok{(}\AttributeTok{mfrow=}\FunctionTok{c}\NormalTok{(}\DecValTok{1}\NormalTok{, }\DecValTok{2}\NormalTok{))}

\NormalTok{util}\SpecialCharTok{$}\FunctionTok{plot\_line\_hist}\NormalTok{(data}\SpecialCharTok{$}\NormalTok{x, }\SpecialCharTok{{-}}\DecValTok{6}\NormalTok{, }\DecValTok{6}\NormalTok{, }\FloatTok{0.5}\NormalTok{,}
                    \AttributeTok{xlab=}\StringTok{"x"}\NormalTok{, }\AttributeTok{main=}\StringTok{"Observed Inputs"}\NormalTok{)}

\NormalTok{util}\SpecialCharTok{$}\FunctionTok{plot\_line\_hist}\NormalTok{(data}\SpecialCharTok{$}\NormalTok{y, }\SpecialCharTok{{-}}\DecValTok{4}\NormalTok{, }\DecValTok{7}\NormalTok{, }\FloatTok{0.5}\NormalTok{,}
                    \AttributeTok{xlab=}\StringTok{"y"}\NormalTok{, }\AttributeTok{main=}\StringTok{"Observed Outputs"}\NormalTok{)}
\end{Highlighting}
\end{Shaded}

\includegraphics{demo_files/figure-pdf/unnamed-chunk-7-1.pdf}

This presentation is a bit cleaner than conventional histogram plots,
especially when there is no ambiguity about the binning. That said the
implementation here is a bit rigid in that it only allows for uniform
bin widths.

A key advantage of reducing a histogram to its outline is that it is
much easier to overlay multiple histograms on top of each other without
compromising legibility. The \texttt{plot\_line\_hists} function
constructs and then overlays two histograms with the same binning.

\begin{Shaded}
\begin{Highlighting}[]
\FunctionTok{par}\NormalTok{(}\AttributeTok{mfrow=}\FunctionTok{c}\NormalTok{(}\DecValTok{1}\NormalTok{, }\DecValTok{1}\NormalTok{))}

\NormalTok{util}\SpecialCharTok{$}\FunctionTok{plot\_line\_hists}\NormalTok{(data}\SpecialCharTok{$}\NormalTok{x, data}\SpecialCharTok{$}\NormalTok{y, }\SpecialCharTok{{-}}\DecValTok{6}\NormalTok{, }\DecValTok{7}\NormalTok{, }\FloatTok{0.5}\NormalTok{, }\AttributeTok{xlab=}\StringTok{""}\NormalTok{)}
\FunctionTok{text}\NormalTok{(}\SpecialCharTok{{-}}\FloatTok{3.5}\NormalTok{, }\DecValTok{90}\NormalTok{, }\AttributeTok{cex=}\FloatTok{1.25}\NormalTok{, }\AttributeTok{label=}\StringTok{"Observed}\SpecialCharTok{\textbackslash{}n}\StringTok{Inputs"}\NormalTok{,}
     \AttributeTok{pos=}\DecValTok{4}\NormalTok{, }\AttributeTok{col=}\StringTok{"black"}\NormalTok{)}
\FunctionTok{text}\NormalTok{(}\FloatTok{3.5}\NormalTok{, }\DecValTok{90}\NormalTok{, }\AttributeTok{cex=}\FloatTok{1.25}\NormalTok{, }\AttributeTok{label=}\StringTok{"Observed}\SpecialCharTok{\textbackslash{}n}\StringTok{Outputs"}\NormalTok{,}
     \AttributeTok{pos=}\DecValTok{4}\NormalTok{, }\AttributeTok{col=}\NormalTok{util}\SpecialCharTok{$}\NormalTok{c\_mid\_teal)}
\end{Highlighting}
\end{Shaded}

\includegraphics{demo_files/figure-pdf/unnamed-chunk-8-1.pdf}

We can also use the \texttt{add} argument of \texttt{plot\_line\_hist}
to overlay multiple histogram outlines onto an existing axis.

\begin{Shaded}
\begin{Highlighting}[]
\FunctionTok{plot}\NormalTok{(}\DecValTok{0}\NormalTok{, }\AttributeTok{type=}\StringTok{"n"}\NormalTok{,}
     \AttributeTok{xlim=}\FunctionTok{c}\NormalTok{(}\SpecialCharTok{{-}}\DecValTok{6}\NormalTok{, }\DecValTok{7}\NormalTok{), }\AttributeTok{xlab=}\StringTok{""}\NormalTok{,}
     \AttributeTok{ylim=}\FunctionTok{c}\NormalTok{(}\DecValTok{0}\NormalTok{, }\DecValTok{160}\NormalTok{), }\AttributeTok{ylab=}\StringTok{"Counts"}\NormalTok{)}

\NormalTok{util}\SpecialCharTok{$}\FunctionTok{plot\_line\_hist}\NormalTok{(data}\SpecialCharTok{$}\NormalTok{x, }\SpecialCharTok{{-}}\DecValTok{6}\NormalTok{, }\DecValTok{6}\NormalTok{, }\FloatTok{0.5}\NormalTok{, }\AttributeTok{col=}\StringTok{"black"}\NormalTok{, }\AttributeTok{add=}\ConstantTok{TRUE}\NormalTok{)}
\NormalTok{util}\SpecialCharTok{$}\FunctionTok{plot\_line\_hist}\NormalTok{(data}\SpecialCharTok{$}\NormalTok{y, }\SpecialCharTok{{-}}\DecValTok{4}\NormalTok{, }\DecValTok{7}\NormalTok{, }\FloatTok{0.5}\NormalTok{, }\AttributeTok{col=}\NormalTok{util}\SpecialCharTok{$}\NormalTok{c\_mid\_teal, }\AttributeTok{add=}\ConstantTok{TRUE}\NormalTok{)}

\FunctionTok{text}\NormalTok{(}\SpecialCharTok{{-}}\FloatTok{3.5}\NormalTok{, }\DecValTok{90}\NormalTok{, }\AttributeTok{cex=}\FloatTok{1.25}\NormalTok{, }\AttributeTok{label=}\StringTok{"Observed}\SpecialCharTok{\textbackslash{}n}\StringTok{Inputs"}\NormalTok{,}
     \AttributeTok{pos=}\DecValTok{4}\NormalTok{, }\AttributeTok{col=}\StringTok{"black"}\NormalTok{)}
\FunctionTok{text}\NormalTok{(}\FloatTok{3.5}\NormalTok{, }\DecValTok{90}\NormalTok{, }\AttributeTok{cex=}\FloatTok{1.25}\NormalTok{, }\AttributeTok{label=}\StringTok{"Observed}\SpecialCharTok{\textbackslash{}n}\StringTok{Outputs"}\NormalTok{,}
     \AttributeTok{pos=}\DecValTok{4}\NormalTok{, }\AttributeTok{col=}\NormalTok{util}\SpecialCharTok{$}\NormalTok{c\_mid\_teal)}
\end{Highlighting}
\end{Shaded}

\includegraphics{demo_files/figure-pdf/unnamed-chunk-9-1.pdf}

\subsection{Prior Checks}\label{prior-checks}

Here we'll be exceptionally thorough and start with an investigation of
the prior model and its consequences. In addition to the individual
parameters we'll look at the prior behavior of baseline function and the
prior predictive distribution along a grid of inputs defined by the
\texttt{x\_grid} array.

\begin{codelisting}

\caption{\texttt{uni\textbackslash\_prior\textbackslash\_model.stan}}

\begin{Shaded}
\begin{Highlighting}[]
\KeywordTok{data}\NormalTok{ \{}
  \DataTypeTok{int}\NormalTok{\textless{}}\KeywordTok{lower}\NormalTok{=}\DecValTok{1}\NormalTok{\textgreater{} N;}
  \DataTypeTok{vector}\NormalTok{[N] x; }\CommentTok{// Observed inputs}
  \DataTypeTok{vector}\NormalTok{[N] y; }\CommentTok{// Observed outputs}

  \DataTypeTok{int}\NormalTok{\textless{}}\KeywordTok{lower}\NormalTok{=}\DecValTok{1}\NormalTok{\textgreater{} N\_grid; }\CommentTok{// Number of grid points for quantifying functional behavior}
  \DataTypeTok{vector}\NormalTok{[N\_grid] x\_grid; }\CommentTok{// Grid points for quantifying functional behavior}
\NormalTok{\}}

\KeywordTok{parameters}\NormalTok{ \{}
  \DataTypeTok{real}\NormalTok{ alpha;}
  \DataTypeTok{real}\NormalTok{ beta;}
  \DataTypeTok{real}\NormalTok{\textless{}}\KeywordTok{lower}\NormalTok{=}\DecValTok{0}\NormalTok{\textgreater{} sigma;}
\NormalTok{\}}

\KeywordTok{model}\NormalTok{ \{}
\NormalTok{  alpha \textasciitilde{} normal(}\DecValTok{0}\NormalTok{, }\DecValTok{3}\NormalTok{ / }\FloatTok{2.32}\NormalTok{);}
\NormalTok{  beta \textasciitilde{} normal(}\DecValTok{0}\NormalTok{, }\DecValTok{3}\NormalTok{ / }\FloatTok{2.32}\NormalTok{);}
\NormalTok{  sigma \textasciitilde{} normal(}\DecValTok{0}\NormalTok{, }\DecValTok{1}\NormalTok{ / }\FloatTok{2.57}\NormalTok{);}
\NormalTok{\}}

\KeywordTok{generated quantities}\NormalTok{ \{}
  \DataTypeTok{vector}\NormalTok{[N\_grid] f\_grid = alpha + beta * x\_grid;}
  \DataTypeTok{array}\NormalTok{[N\_grid] }\DataTypeTok{real}\NormalTok{ y\_pred\_grid = normal\_rng(f\_grid, sigma);}
\NormalTok{\}}
\end{Highlighting}
\end{Shaded}

\end{codelisting}

Note that I'm using a less-aggressive step size adaptation here because
the half-normal prior model for \(\sigma\) results in an slightly
awkward tail for the unconstrained \(\log(\sigma)\) values that can be a
bit difficult to navigate.

\begin{Shaded}
\begin{Highlighting}[]
\NormalTok{data}\SpecialCharTok{$}\NormalTok{N\_grid }\OtherTok{\textless{}{-}} \DecValTok{1000}
\NormalTok{data}\SpecialCharTok{$}\NormalTok{x\_grid }\OtherTok{\textless{}{-}} \FunctionTok{seq}\NormalTok{(}\SpecialCharTok{{-}}\DecValTok{6}\NormalTok{, }\DecValTok{6}\NormalTok{, }\DecValTok{12} \SpecialCharTok{/}\NormalTok{ (data}\SpecialCharTok{$}\NormalTok{N\_grid }\SpecialCharTok{{-}} \DecValTok{1}\NormalTok{))}

\NormalTok{fit }\OtherTok{\textless{}{-}} \FunctionTok{stan}\NormalTok{(}\AttributeTok{file=}\StringTok{\textquotesingle{}stan\_programs/uni\_prior\_model.stan\textquotesingle{}}\NormalTok{,}
            \AttributeTok{data=}\NormalTok{data, }\AttributeTok{seed=}\DecValTok{5838299}\NormalTok{,}
            \AttributeTok{warmup=}\DecValTok{1000}\NormalTok{, }\AttributeTok{iter=}\DecValTok{2024}\NormalTok{, }\AttributeTok{refresh=}\DecValTok{0}\NormalTok{,}
            \AttributeTok{control=}\FunctionTok{list}\NormalTok{(}\StringTok{\textquotesingle{}adapt\_delta\textquotesingle{}} \OtherTok{=} \FloatTok{0.9}\NormalTok{))}
\end{Highlighting}
\end{Shaded}

Of course we always consult our diagnostics first to make sure that our
Markov chains, and hence any visualization we derive from them,
accurately characterize the exact target distribution, in this case the
prior distribution of our model.

\begin{Shaded}
\begin{Highlighting}[]
\NormalTok{diagnostics }\OtherTok{\textless{}{-}}\NormalTok{ util}\SpecialCharTok{$}\FunctionTok{extract\_hmc\_diagnostics}\NormalTok{(fit)}
\NormalTok{util}\SpecialCharTok{$}\FunctionTok{check\_all\_hmc\_diagnostics}\NormalTok{(diagnostics)}
\end{Highlighting}
\end{Shaded}

\begin{verbatim}
  All Hamiltonian Monte Carlo diagnostics are consistent with reliable
Markov chain Monte Carlo.
\end{verbatim}

\begin{Shaded}
\begin{Highlighting}[]
\NormalTok{samples }\OtherTok{\textless{}{-}}\NormalTok{ util}\SpecialCharTok{$}\FunctionTok{extract\_expectand\_vals}\NormalTok{(fit)}
\NormalTok{base\_samples }\OtherTok{\textless{}{-}}\NormalTok{ util}\SpecialCharTok{$}\FunctionTok{filter\_expectands}\NormalTok{(samples,}
                                       \FunctionTok{c}\NormalTok{(}\StringTok{\textquotesingle{}alpha\textquotesingle{}}\NormalTok{, }\StringTok{\textquotesingle{}beta\textquotesingle{}}\NormalTok{, }\StringTok{\textquotesingle{}sigma\textquotesingle{}}\NormalTok{))}
\NormalTok{util}\SpecialCharTok{$}\FunctionTok{check\_all\_expectand\_diagnostics}\NormalTok{(base\_samples)}
\end{Highlighting}
\end{Shaded}

\begin{verbatim}
All expectands checked appear to be behaving well enough for reliable
Markov chain Monte Carlo estimation.
\end{verbatim}

We can visualize the probability distribution of baseline functions in
two ways. Firstly we can plot a subset of baseline function
configurations. Secondly we can plot nested quantile intervals that
quantify the marginal behavior of the function output at each input.
Neither of these visualizations fully characterize the probabilistic
behavior but together they capture the most important features.

The \texttt{plot\_realizations} function plots a selection of values
corresponding to the \texttt{f\_names} array against
\texttt{data\$x\_grid} while the
\texttt{plot\_conn\_pushforward\_quantiles} function plots nested
quantile intervals of those values for each element of
\texttt{data\$x\_grid}. Here ``conn'' refers to ``connected'' as the
individual marginal quantiles are connected into continuous polygons.

\begin{Shaded}
\begin{Highlighting}[]
\FunctionTok{par}\NormalTok{(}\AttributeTok{mfrow=}\FunctionTok{c}\NormalTok{(}\DecValTok{2}\NormalTok{, }\DecValTok{1}\NormalTok{))}

\NormalTok{f\_names }\OtherTok{\textless{}{-}}\NormalTok{ util}\SpecialCharTok{$}\FunctionTok{name\_array}\NormalTok{(}\StringTok{\textquotesingle{}f\_grid\textquotesingle{}}\NormalTok{, }\FunctionTok{c}\NormalTok{(data}\SpecialCharTok{$}\NormalTok{N\_grid))}
\NormalTok{util}\SpecialCharTok{$}\FunctionTok{plot\_realizations}\NormalTok{(samples, f\_names, data}\SpecialCharTok{$}\NormalTok{x\_grid,}
                       \AttributeTok{xlab=}\StringTok{"x"}\NormalTok{, }\AttributeTok{ylab=}\StringTok{"f"}\NormalTok{)}
\NormalTok{util}\SpecialCharTok{$}\FunctionTok{plot\_conn\_pushforward\_quantiles}\NormalTok{(samples, f\_names, data}\SpecialCharTok{$}\NormalTok{x\_grid,}
                                     \AttributeTok{xlab=}\StringTok{"x"}\NormalTok{, }\AttributeTok{ylab=}\StringTok{"f"}\NormalTok{)}
\end{Highlighting}
\end{Shaded}

\includegraphics{demo_files/figure-pdf/unnamed-chunk-12-1.pdf}

Finally let's use the \texttt{plot\_conn\_pushforward\_quantiles}
function to plot nested quantile intervals of the conditional prior
predictive behavior at each element of \texttt{data\$x\_grid}.

\begin{Shaded}
\begin{Highlighting}[]
\FunctionTok{par}\NormalTok{(}\AttributeTok{mfrow=}\FunctionTok{c}\NormalTok{(}\DecValTok{1}\NormalTok{, }\DecValTok{1}\NormalTok{))}

\NormalTok{pred\_names }\OtherTok{\textless{}{-}}\NormalTok{ util}\SpecialCharTok{$}\FunctionTok{name\_array}\NormalTok{(}\StringTok{\textquotesingle{}y\_pred\_grid\textquotesingle{}}\NormalTok{, }\FunctionTok{c}\NormalTok{(data}\SpecialCharTok{$}\NormalTok{N\_grid))}
\NormalTok{util}\SpecialCharTok{$}\FunctionTok{plot\_conn\_pushforward\_quantiles}\NormalTok{(samples, pred\_names, data}\SpecialCharTok{$}\NormalTok{x\_grid,}
                                     \AttributeTok{xlab=}\StringTok{"x"}\NormalTok{, }\AttributeTok{ylab=}\StringTok{"y"}\NormalTok{)}
\end{Highlighting}
\end{Shaded}

\includegraphics{demo_files/figure-pdf/unnamed-chunk-13-1.pdf}

\subsection{Posterior Inference}\label{posterior-inference}

Having thoroughly investigated our prior model and its consequences and
not found any undesired behavior we can move on to constructing
posterior inferences.

\begin{codelisting}

\caption{\texttt{uni\textbackslash\_full\textbackslash\_model.stan}}

\begin{Shaded}
\begin{Highlighting}[]
\KeywordTok{data}\NormalTok{ \{}
  \DataTypeTok{int}\NormalTok{\textless{}}\KeywordTok{lower}\NormalTok{=}\DecValTok{1}\NormalTok{\textgreater{} N;}
  \DataTypeTok{vector}\NormalTok{[N] x; }\CommentTok{// Observed inputs}
  \DataTypeTok{vector}\NormalTok{[N] y; }\CommentTok{// Observed outputs}
  
  \DataTypeTok{int}\NormalTok{\textless{}}\KeywordTok{lower}\NormalTok{=}\DecValTok{1}\NormalTok{\textgreater{} N\_grid; }\CommentTok{// Number of grid points for quantifying functional behavior}
  \DataTypeTok{vector}\NormalTok{[N\_grid] x\_grid; }\CommentTok{// Grid points for quantifying functional behavior}
\NormalTok{\}}

\KeywordTok{parameters}\NormalTok{ \{ }
  \DataTypeTok{real}\NormalTok{ alpha;}
  \DataTypeTok{real}\NormalTok{ beta;}
  \DataTypeTok{real}\NormalTok{\textless{}}\KeywordTok{lower}\NormalTok{=}\DecValTok{0}\NormalTok{\textgreater{} sigma;}
\NormalTok{\}}

\KeywordTok{model}\NormalTok{ \{}
\NormalTok{  alpha \textasciitilde{} normal(}\DecValTok{0}\NormalTok{, }\DecValTok{3}\NormalTok{ / }\FloatTok{2.32}\NormalTok{);}
\NormalTok{  beta \textasciitilde{} normal(}\DecValTok{0}\NormalTok{, }\DecValTok{3}\NormalTok{ / }\FloatTok{2.32}\NormalTok{);}
\NormalTok{  sigma \textasciitilde{} normal(}\DecValTok{0}\NormalTok{, }\DecValTok{1}\NormalTok{ / }\FloatTok{2.57}\NormalTok{);}
  
\NormalTok{  y \textasciitilde{} normal(alpha + beta * x, sigma);}
\NormalTok{\}}

\KeywordTok{generated quantities}\NormalTok{ \{}
  \DataTypeTok{vector}\NormalTok{[N\_grid] f\_grid = alpha + beta * x\_grid;}
  \DataTypeTok{array}\NormalTok{[N\_grid] }\DataTypeTok{real}\NormalTok{ y\_pred\_grid = normal\_rng(f\_grid, sigma);}
\NormalTok{\}}
\end{Highlighting}
\end{Shaded}

\end{codelisting}

\begin{Shaded}
\begin{Highlighting}[]
\NormalTok{fit }\OtherTok{\textless{}{-}} \FunctionTok{stan}\NormalTok{(}\AttributeTok{file=}\StringTok{\textquotesingle{}stan\_programs/uni\_full\_model.stan\textquotesingle{}}\NormalTok{,}
            \AttributeTok{data=}\NormalTok{data, }\AttributeTok{seed=}\DecValTok{5838299}\NormalTok{,}
            \AttributeTok{warmup=}\DecValTok{1000}\NormalTok{, }\AttributeTok{iter=}\DecValTok{2024}\NormalTok{, }\AttributeTok{refresh=}\DecValTok{0}\NormalTok{)}
\end{Highlighting}
\end{Shaded}

There are no signs of trouble from the computational diagnostics.

\begin{Shaded}
\begin{Highlighting}[]
\NormalTok{diagnostics }\OtherTok{\textless{}{-}}\NormalTok{ util}\SpecialCharTok{$}\FunctionTok{extract\_hmc\_diagnostics}\NormalTok{(fit)}
\NormalTok{util}\SpecialCharTok{$}\FunctionTok{check\_all\_hmc\_diagnostics}\NormalTok{(diagnostics)}
\end{Highlighting}
\end{Shaded}

\begin{verbatim}
  All Hamiltonian Monte Carlo diagnostics are consistent with reliable
Markov chain Monte Carlo.
\end{verbatim}

\begin{Shaded}
\begin{Highlighting}[]
\NormalTok{samples }\OtherTok{\textless{}{-}}\NormalTok{ util}\SpecialCharTok{$}\FunctionTok{extract\_expectand\_vals}\NormalTok{(fit)}
\NormalTok{base\_samples }\OtherTok{\textless{}{-}}\NormalTok{ util}\SpecialCharTok{$}\FunctionTok{filter\_expectands}\NormalTok{(samples,}
                                       \FunctionTok{c}\NormalTok{(}\StringTok{\textquotesingle{}alpha\textquotesingle{}}\NormalTok{, }\StringTok{\textquotesingle{}beta\textquotesingle{}}\NormalTok{, }\StringTok{\textquotesingle{}sigma\textquotesingle{}}\NormalTok{))}
\NormalTok{util}\SpecialCharTok{$}\FunctionTok{check\_all\_expectand\_diagnostics}\NormalTok{(base\_samples)}
\end{Highlighting}
\end{Shaded}

\begin{verbatim}
All expectands checked appear to be behaving well enough for reliable
Markov chain Monte Carlo estimation.
\end{verbatim}

Before examining any posterior inferences, however, we need to validate
that our model is adequately capturing the relevant features of the
observed data. For this one-dimensional baseline function model we can
implement an informative retrodictive check by comparing the conditional
posterior predictive distributions at each input, \[
p(y \mid x,
         \tilde{x}_{1}, \tilde{y}_{1}, \ldots, \tilde{x}_{N}, \tilde{y}_{N}),
\] to the observed input-output pairs, \((x_{n}, y_{n})\).

\begin{Shaded}
\begin{Highlighting}[]
\FunctionTok{par}\NormalTok{(}\AttributeTok{mfrow=}\FunctionTok{c}\NormalTok{(}\DecValTok{1}\NormalTok{, }\DecValTok{1}\NormalTok{))}

\NormalTok{pred\_names }\OtherTok{\textless{}{-}}\NormalTok{ util}\SpecialCharTok{$}\FunctionTok{name\_array}\NormalTok{(}\StringTok{\textquotesingle{}y\_pred\_grid\textquotesingle{}}\NormalTok{, }\FunctionTok{c}\NormalTok{(data}\SpecialCharTok{$}\NormalTok{N\_grid))}
\NormalTok{util}\SpecialCharTok{$}\FunctionTok{plot\_conn\_pushforward\_quantiles}\NormalTok{(samples, pred\_names, data}\SpecialCharTok{$}\NormalTok{x\_grid,}
                                     \AttributeTok{xlab=}\StringTok{"x"}\NormalTok{, }\AttributeTok{ylab=}\StringTok{"y"}\NormalTok{)}
\FunctionTok{points}\NormalTok{(data}\SpecialCharTok{$}\NormalTok{x, data}\SpecialCharTok{$}\NormalTok{y, }\AttributeTok{pch=}\DecValTok{16}\NormalTok{, }\AttributeTok{cex=}\FloatTok{1.0}\NormalTok{, }\AttributeTok{col=}\StringTok{"white"}\NormalTok{)}
\FunctionTok{points}\NormalTok{(data}\SpecialCharTok{$}\NormalTok{x, data}\SpecialCharTok{$}\NormalTok{y, }\AttributeTok{pch=}\DecValTok{16}\NormalTok{, }\AttributeTok{cex=}\FloatTok{0.8}\NormalTok{, }\AttributeTok{col=}\StringTok{"black"}\NormalTok{)}
\end{Highlighting}
\end{Shaded}

\includegraphics{demo_files/figure-pdf/unnamed-chunk-16-1.pdf}

Fortunately there are no signs of tension between the posterior
predictive distributional behaviors and the observed behaviors.
Confident in the adequacy of our model we can move onto visualizing
posterior inferences.

For example we can visualize the pushforward, or marginal, probability
distributions for each parameter. Note that the
\texttt{plot\_expectand\_pushforward} function is already part of my
Markov chain Monte Carlo analysis tools and not one of the visualization
functions being introduced here.

\begin{Shaded}
\begin{Highlighting}[]
\FunctionTok{par}\NormalTok{(}\AttributeTok{mfrow=}\FunctionTok{c}\NormalTok{(}\DecValTok{1}\NormalTok{, }\DecValTok{3}\NormalTok{))}

\NormalTok{util}\SpecialCharTok{$}\FunctionTok{plot\_expectand\_pushforward}\NormalTok{(samples[[}\StringTok{"alpha"}\NormalTok{]], }\DecValTok{25}\NormalTok{,}
                                \AttributeTok{display\_name=}\StringTok{"alpha"}\NormalTok{)}

\NormalTok{util}\SpecialCharTok{$}\FunctionTok{plot\_expectand\_pushforward}\NormalTok{(samples[[}\StringTok{"beta"}\NormalTok{]], }\DecValTok{25}\NormalTok{,}
                                \AttributeTok{display\_name=}\StringTok{"beta"}\NormalTok{)}

\NormalTok{util}\SpecialCharTok{$}\FunctionTok{plot\_expectand\_pushforward}\NormalTok{(samples[[}\StringTok{"sigma"}\NormalTok{]], }\DecValTok{25}\NormalTok{,}
                                \AttributeTok{display\_name=}\StringTok{"sigma"}\NormalTok{)}
\end{Highlighting}
\end{Shaded}

\includegraphics{demo_files/figure-pdf/unnamed-chunk-17-1.pdf}

Communicating the posterior behavior of the baseline function, however,
is facilitated with the new visualization functions.

\begin{Shaded}
\begin{Highlighting}[]
\FunctionTok{par}\NormalTok{(}\AttributeTok{mfrow=}\FunctionTok{c}\NormalTok{(}\DecValTok{2}\NormalTok{, }\DecValTok{1}\NormalTok{))}

\NormalTok{f\_names }\OtherTok{\textless{}{-}}\NormalTok{ util}\SpecialCharTok{$}\FunctionTok{name\_array}\NormalTok{(}\StringTok{\textquotesingle{}f\_grid\textquotesingle{}}\NormalTok{, }\FunctionTok{c}\NormalTok{(data}\SpecialCharTok{$}\NormalTok{N\_grid))}
\NormalTok{util}\SpecialCharTok{$}\FunctionTok{plot\_realizations}\NormalTok{(samples, f\_names, data}\SpecialCharTok{$}\NormalTok{x\_grid,}
                       \AttributeTok{xlab=}\StringTok{"x"}\NormalTok{, }\AttributeTok{ylab=}\StringTok{"f"}\NormalTok{)}
\NormalTok{util}\SpecialCharTok{$}\FunctionTok{plot\_conn\_pushforward\_quantiles}\NormalTok{(samples, f\_names, data}\SpecialCharTok{$}\NormalTok{x\_grid,}
                                     \AttributeTok{xlab=}\StringTok{"x"}\NormalTok{, }\AttributeTok{ylab=}\StringTok{"f"}\NormalTok{)}
\end{Highlighting}
\end{Shaded}

\includegraphics{demo_files/figure-pdf/unnamed-chunk-18-1.pdf}

Conveniently all of these visualization functions feature optional
arguments for baseline behavior which allows us to compare our posterior
inferences to the true behavior when it is known, for example in
simulation studies.

\begin{Shaded}
\begin{Highlighting}[]
\NormalTok{true\_alpha }\OtherTok{\textless{}{-}} \FloatTok{1.5}
\NormalTok{true\_beta }\OtherTok{\textless{}{-}} \SpecialCharTok{{-}}\FloatTok{0.75}
\NormalTok{true\_sigma }\OtherTok{\textless{}{-}} \FloatTok{0.33}

\FunctionTok{par}\NormalTok{(}\AttributeTok{mfrow=}\FunctionTok{c}\NormalTok{(}\DecValTok{1}\NormalTok{, }\DecValTok{3}\NormalTok{))}

\NormalTok{util}\SpecialCharTok{$}\FunctionTok{plot\_expectand\_pushforward}\NormalTok{(samples[[}\StringTok{"alpha"}\NormalTok{]], }\DecValTok{25}\NormalTok{,}
                                \AttributeTok{display\_name=}\StringTok{"alpha"}\NormalTok{,}
                                \AttributeTok{baseline=}\NormalTok{true\_alpha,}
                                \AttributeTok{baseline\_col=}\NormalTok{util}\SpecialCharTok{$}\NormalTok{c\_mid\_teal)}

\NormalTok{util}\SpecialCharTok{$}\FunctionTok{plot\_expectand\_pushforward}\NormalTok{(samples[[}\StringTok{"beta"}\NormalTok{]], }\DecValTok{25}\NormalTok{,}
                                \AttributeTok{display\_name=}\StringTok{"beta"}\NormalTok{,}
                                \AttributeTok{baseline=}\NormalTok{true\_beta,}
                                \AttributeTok{baseline\_col=}\NormalTok{util}\SpecialCharTok{$}\NormalTok{c\_mid\_teal)}

\NormalTok{util}\SpecialCharTok{$}\FunctionTok{plot\_expectand\_pushforward}\NormalTok{(samples[[}\StringTok{"sigma"}\NormalTok{]], }\DecValTok{25}\NormalTok{,}
                                \AttributeTok{display\_name=}\StringTok{"sigma"}\NormalTok{,}
                                \AttributeTok{baseline=}\NormalTok{true\_sigma,}
                                \AttributeTok{baseline\_col=}\NormalTok{util}\SpecialCharTok{$}\NormalTok{c\_mid\_teal)}
\end{Highlighting}
\end{Shaded}

\includegraphics{demo_files/figure-pdf/unnamed-chunk-19-1.pdf}

\begin{Shaded}
\begin{Highlighting}[]
\NormalTok{true\_fs }\OtherTok{\textless{}{-}}\NormalTok{ true\_alpha }\SpecialCharTok{+}\NormalTok{ true\_beta }\SpecialCharTok{*}\NormalTok{ data}\SpecialCharTok{$}\NormalTok{x\_grid}

\FunctionTok{par}\NormalTok{(}\AttributeTok{mfrow=}\FunctionTok{c}\NormalTok{(}\DecValTok{2}\NormalTok{, }\DecValTok{1}\NormalTok{))}

\NormalTok{util}\SpecialCharTok{$}\FunctionTok{plot\_realizations}\NormalTok{(samples, f\_names, data}\SpecialCharTok{$}\NormalTok{x\_grid,}
                       \AttributeTok{baseline\_values=}\NormalTok{true\_fs,}
                       \AttributeTok{baseline\_col=}\NormalTok{util}\SpecialCharTok{$}\NormalTok{c\_mid\_teal,}
                       \AttributeTok{xlab=}\StringTok{"x"}\NormalTok{, }\AttributeTok{ylab=}\StringTok{"f"}\NormalTok{)}
\NormalTok{util}\SpecialCharTok{$}\FunctionTok{plot\_conn\_pushforward\_quantiles}\NormalTok{(samples, f\_names, data}\SpecialCharTok{$}\NormalTok{x\_grid,}
                                     \AttributeTok{baseline\_values=}\NormalTok{true\_fs,}
                                     \AttributeTok{baseline\_col=}\NormalTok{util}\SpecialCharTok{$}\NormalTok{c\_mid\_teal,}
                                     \AttributeTok{xlab=}\StringTok{"x"}\NormalTok{, }\AttributeTok{ylab=}\StringTok{"f"}\NormalTok{)}
\end{Highlighting}
\end{Shaded}

\includegraphics{demo_files/figure-pdf/unnamed-chunk-20-1.pdf}

The \texttt{plot\_realizations} and
\texttt{plot\_conn\_pushforward\_quantiles} functions also include
\texttt{residual} arguments that allow us to directly visualize how the
probabilistic behavior varies around the baseline values.

\begin{Shaded}
\begin{Highlighting}[]
\FunctionTok{par}\NormalTok{(}\AttributeTok{mfrow=}\FunctionTok{c}\NormalTok{(}\DecValTok{2}\NormalTok{, }\DecValTok{1}\NormalTok{))}

\NormalTok{util}\SpecialCharTok{$}\FunctionTok{plot\_realizations}\NormalTok{(samples, f\_names, data}\SpecialCharTok{$}\NormalTok{x\_grid,}
                       \AttributeTok{baseline\_values=}\NormalTok{true\_fs,}
                       \AttributeTok{residual=}\ConstantTok{TRUE}\NormalTok{,}
                       \AttributeTok{xlab=}\StringTok{"x"}\NormalTok{, }\AttributeTok{ylab=}\StringTok{"f"}\NormalTok{)}
\NormalTok{util}\SpecialCharTok{$}\FunctionTok{plot\_conn\_pushforward\_quantiles}\NormalTok{(samples, f\_names, data}\SpecialCharTok{$}\NormalTok{x\_grid,}
                                     \AttributeTok{baseline\_values=}\NormalTok{true\_fs,}
                                     \AttributeTok{residual=}\ConstantTok{TRUE}\NormalTok{,}
                                     \AttributeTok{xlab=}\StringTok{"x"}\NormalTok{, }\AttributeTok{ylab=}\StringTok{"f"}\NormalTok{)}
\end{Highlighting}
\end{Shaded}

\includegraphics{demo_files/figure-pdf/unnamed-chunk-21-1.pdf}

\section{Multi-Dimensional Baseline
Function}\label{multi-dimensional-baseline-function}

Now that we're warmed up let's consider a three-dimensional
curve-fitting model with a quadratic baseline function, \[
p(y_{n} \mid \mathbf{x}_{n}, \alpha, \beta, \sigma)
=
\text{normal}(y_{n}   \mid \beta_{0}
                    + \boldsymbol{\beta}^{T} \cdot \mathbf{x}
                    + \mathbf{x}^{T} \cdot \mathbf{B} \cdot \mathbf{x}, \sigma),
\] where \(\mathbf{B}\) is a positive-definite matrix whose three
diagonal elements are organized into the vector
\(\boldsymbol{\beta}_{d}\) and three off-diagonal elements are organized
into the vector \(\boldsymbol{\beta}_{o}\).

\subsection{Plot Data}\label{plot-data}

The \texttt{plot\_line\_hist} allows us to cleanly visualize each
component of the observed inputs.

\begin{Shaded}
\begin{Highlighting}[]
\NormalTok{data }\OtherTok{\textless{}{-}} \FunctionTok{read\_rdump}\NormalTok{(}\StringTok{"data/multi\_data.R"}\NormalTok{)}

\FunctionTok{par}\NormalTok{(}\AttributeTok{mfrow=}\FunctionTok{c}\NormalTok{(}\DecValTok{3}\NormalTok{, }\DecValTok{3}\NormalTok{), }\AttributeTok{mar =} \FunctionTok{c}\NormalTok{(}\DecValTok{5}\NormalTok{, }\DecValTok{5}\NormalTok{, }\DecValTok{2}\NormalTok{, }\DecValTok{1}\NormalTok{))}

\NormalTok{util}\SpecialCharTok{$}\FunctionTok{plot\_line\_hist}\NormalTok{(data}\SpecialCharTok{$}\NormalTok{X[,}\DecValTok{1}\NormalTok{], }\SpecialCharTok{{-}}\DecValTok{9}\NormalTok{, }\DecValTok{9}\NormalTok{, }\DecValTok{1}\NormalTok{, }\AttributeTok{xlab=}\StringTok{"x1"}\NormalTok{)}
\NormalTok{util}\SpecialCharTok{$}\FunctionTok{plot\_line\_hist}\NormalTok{(data}\SpecialCharTok{$}\NormalTok{X[,}\DecValTok{2}\NormalTok{], }\SpecialCharTok{{-}}\DecValTok{9}\NormalTok{, }\DecValTok{9}\NormalTok{, }\DecValTok{1}\NormalTok{, }\AttributeTok{xlab=}\StringTok{"x1"}\NormalTok{)}
\NormalTok{util}\SpecialCharTok{$}\FunctionTok{plot\_line\_hist}\NormalTok{(data}\SpecialCharTok{$}\NormalTok{X[,}\DecValTok{3}\NormalTok{], }\SpecialCharTok{{-}}\DecValTok{9}\NormalTok{, }\DecValTok{9}\NormalTok{, }\DecValTok{1}\NormalTok{, }\AttributeTok{xlab=}\StringTok{"x1"}\NormalTok{)}

\FunctionTok{plot}\NormalTok{(data}\SpecialCharTok{$}\NormalTok{X[,}\DecValTok{1}\NormalTok{], data}\SpecialCharTok{$}\NormalTok{X[,}\DecValTok{2}\NormalTok{], }\AttributeTok{pch=}\DecValTok{16}\NormalTok{, }\AttributeTok{cex=}\FloatTok{1.0}\NormalTok{, }\AttributeTok{col=}\StringTok{"black"}\NormalTok{,}
     \AttributeTok{main=}\StringTok{""}\NormalTok{, }\AttributeTok{xlim=}\FunctionTok{c}\NormalTok{(}\SpecialCharTok{{-}}\DecValTok{9}\NormalTok{, }\DecValTok{9}\NormalTok{), }\AttributeTok{xlab=}\StringTok{"x1"}\NormalTok{, }\AttributeTok{ylim=}\FunctionTok{c}\NormalTok{(}\SpecialCharTok{{-}}\DecValTok{9}\NormalTok{, }\DecValTok{9}\NormalTok{), }\AttributeTok{ylab=}\StringTok{"x2"}\NormalTok{)}

\FunctionTok{plot}\NormalTok{(data}\SpecialCharTok{$}\NormalTok{X[,}\DecValTok{1}\NormalTok{], data}\SpecialCharTok{$}\NormalTok{X[,}\DecValTok{3}\NormalTok{], }\AttributeTok{pch=}\DecValTok{16}\NormalTok{, }\AttributeTok{cex=}\FloatTok{1.0}\NormalTok{, }\AttributeTok{col=}\StringTok{"black"}\NormalTok{,}
     \AttributeTok{main=}\StringTok{""}\NormalTok{, }\AttributeTok{xlim=}\FunctionTok{c}\NormalTok{(}\SpecialCharTok{{-}}\DecValTok{9}\NormalTok{, }\DecValTok{9}\NormalTok{), }\AttributeTok{xlab=}\StringTok{"x1"}\NormalTok{, }\AttributeTok{ylim=}\FunctionTok{c}\NormalTok{(}\SpecialCharTok{{-}}\DecValTok{9}\NormalTok{, }\DecValTok{9}\NormalTok{), }\AttributeTok{ylab=}\StringTok{"x3"}\NormalTok{)}

\FunctionTok{plot}\NormalTok{(data}\SpecialCharTok{$}\NormalTok{X[,}\DecValTok{2}\NormalTok{], data}\SpecialCharTok{$}\NormalTok{X[,}\DecValTok{3}\NormalTok{], }\AttributeTok{pch=}\DecValTok{16}\NormalTok{, }\AttributeTok{cex=}\FloatTok{1.0}\NormalTok{, }\AttributeTok{col=}\StringTok{"black"}\NormalTok{,}
     \AttributeTok{main=}\StringTok{""}\NormalTok{, }\AttributeTok{xlim=}\FunctionTok{c}\NormalTok{(}\SpecialCharTok{{-}}\DecValTok{9}\NormalTok{, }\DecValTok{9}\NormalTok{), }\AttributeTok{xlab=}\StringTok{"x2"}\NormalTok{, }\AttributeTok{ylim=}\FunctionTok{c}\NormalTok{(}\SpecialCharTok{{-}}\DecValTok{9}\NormalTok{, }\DecValTok{9}\NormalTok{), }\AttributeTok{ylab=}\StringTok{"x3"}\NormalTok{)}

\FunctionTok{plot}\NormalTok{(data}\SpecialCharTok{$}\NormalTok{X[,}\DecValTok{1}\NormalTok{], data}\SpecialCharTok{$}\NormalTok{y, }\AttributeTok{pch=}\DecValTok{16}\NormalTok{, }\AttributeTok{cex=}\FloatTok{1.0}\NormalTok{, }\AttributeTok{col=}\StringTok{"black"}\NormalTok{,}
     \AttributeTok{main=}\StringTok{""}\NormalTok{, }\AttributeTok{xlim=}\FunctionTok{c}\NormalTok{(}\SpecialCharTok{{-}}\DecValTok{9}\NormalTok{, }\DecValTok{9}\NormalTok{), }\AttributeTok{xlab=}\StringTok{"x1"}\NormalTok{, }\AttributeTok{ylim=}\FunctionTok{c}\NormalTok{(}\SpecialCharTok{{-}}\DecValTok{25}\NormalTok{, }\DecValTok{325}\NormalTok{), }\AttributeTok{ylab=}\StringTok{"y"}\NormalTok{)}

\FunctionTok{plot}\NormalTok{(data}\SpecialCharTok{$}\NormalTok{X[,}\DecValTok{2}\NormalTok{], data}\SpecialCharTok{$}\NormalTok{y, }\AttributeTok{pch=}\DecValTok{16}\NormalTok{, }\AttributeTok{cex=}\FloatTok{1.0}\NormalTok{, }\AttributeTok{col=}\StringTok{"black"}\NormalTok{,}
     \AttributeTok{main=}\StringTok{""}\NormalTok{, }\AttributeTok{xlim=}\FunctionTok{c}\NormalTok{(}\SpecialCharTok{{-}}\DecValTok{9}\NormalTok{, }\DecValTok{9}\NormalTok{), }\AttributeTok{xlab=}\StringTok{"x2"}\NormalTok{, }\AttributeTok{ylim=}\FunctionTok{c}\NormalTok{(}\SpecialCharTok{{-}}\DecValTok{25}\NormalTok{, }\DecValTok{325}\NormalTok{), }\AttributeTok{ylab=}\StringTok{"y"}\NormalTok{)}

\FunctionTok{plot}\NormalTok{(data}\SpecialCharTok{$}\NormalTok{X[,}\DecValTok{3}\NormalTok{], data}\SpecialCharTok{$}\NormalTok{y, }\AttributeTok{pch=}\DecValTok{16}\NormalTok{, }\AttributeTok{cex=}\FloatTok{1.0}\NormalTok{, }\AttributeTok{col=}\StringTok{"black"}\NormalTok{,}
     \AttributeTok{main=}\StringTok{""}\NormalTok{, }\AttributeTok{xlim=}\FunctionTok{c}\NormalTok{(}\SpecialCharTok{{-}}\DecValTok{9}\NormalTok{, }\DecValTok{9}\NormalTok{), }\AttributeTok{xlab=}\StringTok{"x3"}\NormalTok{, }\AttributeTok{ylim=}\FunctionTok{c}\NormalTok{(}\SpecialCharTok{{-}}\DecValTok{25}\NormalTok{, }\DecValTok{325}\NormalTok{), }\AttributeTok{ylab=}\StringTok{"y"}\NormalTok{)}
\end{Highlighting}
\end{Shaded}

\includegraphics{demo_files/figure-pdf/unnamed-chunk-22-1.pdf}

\subsection{Prior Checks}\label{prior-checks-1}

As before we'll first investigate the consequences of our prior model.

For a discussion of why the quadratic baseline model is implemented in
this way see Section 2.3.2 of my
\href{https://betanalpha.github.io/assets/case_studies/taylor_models.html\#232_Higher-Order_Implementations}{Taylor
regression modeling chapter}.

\begin{codelisting}

\caption{\texttt{multi\textbackslash\_prior\textbackslash\_model.stan}}

\begin{Shaded}
\begin{Highlighting}[]
\KeywordTok{data}\NormalTok{ \{}
  \DataTypeTok{int}\NormalTok{\textless{}}\KeywordTok{lower}\NormalTok{=}\DecValTok{0}\NormalTok{\textgreater{} M;  }\CommentTok{// Number of covariates}
  \DataTypeTok{int}\NormalTok{\textless{}}\KeywordTok{lower}\NormalTok{=}\DecValTok{0}\NormalTok{\textgreater{} N;  }\CommentTok{// Number of observations}
  
  \DataTypeTok{vector}\NormalTok{[M] x0;    }\CommentTok{// Covariate baselines}
  \DataTypeTok{matrix}\NormalTok{[N, M] X;  }\CommentTok{// Covariate design matrix}
\NormalTok{\}}

\KeywordTok{transformed data}\NormalTok{ \{}
  \DataTypeTok{matrix}\NormalTok{[N, M * (M + }\DecValTok{3}\NormalTok{) / }\DecValTok{2}\NormalTok{ + }\DecValTok{1}\NormalTok{] deltaX;}
  \ControlFlowTok{for}\NormalTok{ (n }\ControlFlowTok{in} \DecValTok{1}\NormalTok{:N) \{}
\NormalTok{    deltaX[n, }\DecValTok{1}\NormalTok{] = }\DecValTok{1}\NormalTok{;}
    
    \ControlFlowTok{for}\NormalTok{ (m1 }\ControlFlowTok{in} \DecValTok{1}\NormalTok{:M) \{}
      \CommentTok{// Linear perturbations}
\NormalTok{      deltaX[n, m1 + }\DecValTok{1}\NormalTok{] = X[n, m1] {-} x0[m1];}
\NormalTok{    \}}
    
    \ControlFlowTok{for}\NormalTok{ (m1 }\ControlFlowTok{in} \DecValTok{1}\NormalTok{:M) \{}
      \CommentTok{// On{-}diagonal quadratic perturbations}
\NormalTok{      deltaX[n, M + m1 + }\DecValTok{1}\NormalTok{] }
\NormalTok{        = deltaX[n, m1 + }\DecValTok{1}\NormalTok{] * deltaX[n, m1 + }\DecValTok{1}\NormalTok{];}
  
      \ControlFlowTok{for}\NormalTok{ (m2 }\ControlFlowTok{in}\NormalTok{ (m1 + }\DecValTok{1}\NormalTok{):M) \{}
        \DataTypeTok{int}\NormalTok{ m3 = (}\DecValTok{2}\NormalTok{ * M {-} m1) * (m1 {-} }\DecValTok{1}\NormalTok{) / }\DecValTok{2}\NormalTok{ + m2 {-} m1;}
          
        \CommentTok{// Off{-}diagonal quadratic perturbations}
        \CommentTok{// Factor of 2 ensures that beta parameters have the}
        \CommentTok{// same interpretation as the expanded implementation}
\NormalTok{        deltaX[n, }\DecValTok{2}\NormalTok{ * M + m3 + }\DecValTok{1}\NormalTok{] }
\NormalTok{          = }\DecValTok{2}\NormalTok{ * deltaX[n, m1 + }\DecValTok{1}\NormalTok{] * deltaX[n, m2 + }\DecValTok{1}\NormalTok{];}
\NormalTok{      \}}
\NormalTok{    \}}
\NormalTok{  \}}
\NormalTok{\}}

\KeywordTok{parameters}\NormalTok{ \{}
  \DataTypeTok{real}\NormalTok{ beta0;                      }\CommentTok{// Intercept}
  \DataTypeTok{vector}\NormalTok{[M] beta1;                 }\CommentTok{// Linear slopes}
  \DataTypeTok{vector}\NormalTok{[M] beta2\_d;               }\CommentTok{// On{-}diagonal quadratic slopes}
  \DataTypeTok{vector}\NormalTok{[M * (M {-} }\DecValTok{1}\NormalTok{) / }\DecValTok{2}\NormalTok{] beta2\_o; }\CommentTok{// Off{-}diagonal quadratic slopes}
  \DataTypeTok{real}\NormalTok{\textless{}}\KeywordTok{lower}\NormalTok{=}\DecValTok{0}\NormalTok{\textgreater{} sigma;             }\CommentTok{// Measurement Variability}
\NormalTok{\}}

\KeywordTok{model}\NormalTok{ \{}
  \DataTypeTok{vector}\NormalTok{[M * (M + }\DecValTok{3}\NormalTok{) / }\DecValTok{2}\NormalTok{ + }\DecValTok{1}\NormalTok{] beta}
\NormalTok{    = append\_row(}
\NormalTok{        append\_row(}
\NormalTok{          append\_row(beta0, beta1), }
\NormalTok{        beta2\_d), }
\NormalTok{      beta2\_o);}
  
  \CommentTok{// Prior model}
\NormalTok{  beta0 \textasciitilde{} normal(}\DecValTok{0}\NormalTok{, }\DecValTok{10}\NormalTok{);}
\NormalTok{  beta1 \textasciitilde{} normal(}\DecValTok{0}\NormalTok{, }\DecValTok{10}\NormalTok{);}
\NormalTok{  beta2\_d \textasciitilde{} normal(}\DecValTok{0}\NormalTok{, }\DecValTok{2}\NormalTok{);}
\NormalTok{  beta2\_o \textasciitilde{} normal(}\DecValTok{0}\NormalTok{, }\DecValTok{1}\NormalTok{);}
\NormalTok{  sigma \textasciitilde{} normal(}\DecValTok{0}\NormalTok{, }\DecValTok{5}\NormalTok{);}
\NormalTok{\}}

\KeywordTok{generated quantities}\NormalTok{ \{}
  \CommentTok{// Posterior predictions}
  \DataTypeTok{array}\NormalTok{[N] }\DataTypeTok{real}\NormalTok{ y\_pred;}
\NormalTok{  \{}
    \DataTypeTok{vector}\NormalTok{[M * (M + }\DecValTok{3}\NormalTok{) / }\DecValTok{2}\NormalTok{ + }\DecValTok{1}\NormalTok{] beta}
\NormalTok{      = append\_row(}
\NormalTok{          append\_row(}
\NormalTok{            append\_row(beta0, beta1), }
\NormalTok{          beta2\_d),}
\NormalTok{        beta2\_o);}
\NormalTok{    y\_pred = normal\_rng(deltaX * beta, sigma);}
\NormalTok{  \}}
\NormalTok{\}}
\end{Highlighting}
\end{Shaded}

\end{codelisting}

\begin{Shaded}
\begin{Highlighting}[]
\NormalTok{fit }\OtherTok{\textless{}{-}} \FunctionTok{stan}\NormalTok{(}\AttributeTok{file=}\StringTok{\textquotesingle{}stan\_programs/multi\_prior\_model.stan\textquotesingle{}}\NormalTok{,}
            \AttributeTok{data=}\NormalTok{data, }\AttributeTok{seed=}\DecValTok{5838299}\NormalTok{,}
            \AttributeTok{warmup=}\DecValTok{1000}\NormalTok{, }\AttributeTok{iter=}\DecValTok{2024}\NormalTok{, }\AttributeTok{refresh=}\DecValTok{0}\NormalTok{)}
\end{Highlighting}
\end{Shaded}

Higher-dimensional probability distributions are no trouble for
Hamiltonian Monte Carlo.

\begin{Shaded}
\begin{Highlighting}[]
\NormalTok{diagnostics }\OtherTok{\textless{}{-}}\NormalTok{ util}\SpecialCharTok{$}\FunctionTok{extract\_hmc\_diagnostics}\NormalTok{(fit)}
\NormalTok{util}\SpecialCharTok{$}\FunctionTok{check\_all\_hmc\_diagnostics}\NormalTok{(diagnostics)}
\end{Highlighting}
\end{Shaded}

\begin{verbatim}
  All Hamiltonian Monte Carlo diagnostics are consistent with reliable
Markov chain Monte Carlo.
\end{verbatim}

\begin{Shaded}
\begin{Highlighting}[]
\NormalTok{samples }\OtherTok{\textless{}{-}}\NormalTok{ util}\SpecialCharTok{$}\FunctionTok{extract\_expectand\_vals}\NormalTok{(fit)}
\NormalTok{base\_samples }\OtherTok{\textless{}{-}}\NormalTok{ util}\SpecialCharTok{$}\FunctionTok{filter\_expectands}\NormalTok{(samples,}
                                       \FunctionTok{c}\NormalTok{(}\StringTok{\textquotesingle{}beta0\textquotesingle{}}\NormalTok{, }\StringTok{\textquotesingle{}beta1\textquotesingle{}}\NormalTok{,}
                                         \StringTok{\textquotesingle{}beta2\_d\textquotesingle{}}\NormalTok{, }\StringTok{\textquotesingle{}beta2\_o\textquotesingle{}}\NormalTok{,}
                                         \StringTok{\textquotesingle{}sigma\textquotesingle{}}\NormalTok{),}
                                       \ConstantTok{TRUE}\NormalTok{)}
\NormalTok{util}\SpecialCharTok{$}\FunctionTok{check\_all\_expectand\_diagnostics}\NormalTok{(base\_samples)}
\end{Highlighting}
\end{Shaded}

\begin{verbatim}
All expectands checked appear to be behaving well enough for reliable
Markov chain Monte Carlo estimation.
\end{verbatim}

With a multi-dimensional input space we can no longer visualize the
baseline functional behavior nor the conditional prior predictive
behavior directly. We can, however, visualize many of its features.

For example we might consider the marginal behavior of the predicted
outputs, regardless of the corresponding observed inputs. Here we'll
summarize this marginal behavior with a histogram, and use the
\texttt{plot\_hist\_quantiles} function to visualize the prior
predictive distribution of the histogram counts.

\begin{Shaded}
\begin{Highlighting}[]
\FunctionTok{par}\NormalTok{(}\AttributeTok{mfrow=}\FunctionTok{c}\NormalTok{(}\DecValTok{1}\NormalTok{, }\DecValTok{1}\NormalTok{))}

\NormalTok{util}\SpecialCharTok{$}\FunctionTok{plot\_hist\_quantiles}\NormalTok{(samples, }\StringTok{\textquotesingle{}y\_pred\textquotesingle{}}\NormalTok{)}
\end{Highlighting}
\end{Shaded}

\includegraphics{demo_files/figure-pdf/unnamed-chunk-25-1.pdf}

To capture the interactions between the predictive outputs and the
observed input components we'll need a more sophisticated summary
statistic. Here we'll use the empirical mean and medians of the
predictive outputs within bins of each input component. For a detailed
discussion of how this summary statistic is constructed see Section 2.5
of my
\href{https://betanalpha.github.io/assets/case_studies/taylor_models.html\#25_Posterior_Retrodictive_Checks}{Taylor
regression modeling chapter}.

Conveniently the \texttt{plot\_conditional\_mean\_quantiles} and
\texttt{plot\_conditional\_median\_quantiles} functions visualize the
prior predictive behavior of these summary statistics.

\begin{Shaded}
\begin{Highlighting}[]
\FunctionTok{par}\NormalTok{(}\AttributeTok{mfrow=}\FunctionTok{c}\NormalTok{(}\DecValTok{1}\NormalTok{, }\DecValTok{3}\NormalTok{))}

\NormalTok{pred\_names }\OtherTok{\textless{}{-}}\NormalTok{ util}\SpecialCharTok{$}\FunctionTok{name\_array}\NormalTok{(}\StringTok{\textquotesingle{}y\_pred\textquotesingle{}}\NormalTok{, }\FunctionTok{c}\NormalTok{(data}\SpecialCharTok{$}\NormalTok{N))}
\NormalTok{util}\SpecialCharTok{$}\FunctionTok{plot\_conditional\_mean\_quantiles}\NormalTok{(samples, pred\_names, data}\SpecialCharTok{$}\NormalTok{X[,}\DecValTok{1}\NormalTok{],}
                                     \SpecialCharTok{{-}}\DecValTok{9}\NormalTok{, }\DecValTok{9}\NormalTok{, }\DecValTok{1}\NormalTok{, }\AttributeTok{xlab=}\StringTok{"x1"}\NormalTok{)}
\NormalTok{util}\SpecialCharTok{$}\FunctionTok{plot\_conditional\_mean\_quantiles}\NormalTok{(samples, pred\_names, data}\SpecialCharTok{$}\NormalTok{X[,}\DecValTok{2}\NormalTok{],}
                                     \SpecialCharTok{{-}}\DecValTok{9}\NormalTok{, }\DecValTok{9}\NormalTok{, }\DecValTok{1}\NormalTok{,}\AttributeTok{xlab=}\StringTok{"x2"}\NormalTok{)}
\NormalTok{util}\SpecialCharTok{$}\FunctionTok{plot\_conditional\_mean\_quantiles}\NormalTok{(samples, pred\_names, data}\SpecialCharTok{$}\NormalTok{X[,}\DecValTok{3}\NormalTok{],}
                                     \SpecialCharTok{{-}}\DecValTok{9}\NormalTok{, }\DecValTok{9}\NormalTok{, }\DecValTok{1}\NormalTok{, }\AttributeTok{xlab=}\StringTok{"x3"}\NormalTok{)}
\end{Highlighting}
\end{Shaded}

\includegraphics{demo_files/figure-pdf/unnamed-chunk-26-1.pdf}

\begin{Shaded}
\begin{Highlighting}[]
\FunctionTok{par}\NormalTok{(}\AttributeTok{mfrow=}\FunctionTok{c}\NormalTok{(}\DecValTok{1}\NormalTok{, }\DecValTok{3}\NormalTok{))}

\NormalTok{util}\SpecialCharTok{$}\FunctionTok{plot\_conditional\_median\_quantiles}\NormalTok{(samples, pred\_names, data}\SpecialCharTok{$}\NormalTok{X[,}\DecValTok{1}\NormalTok{],}
                                       \SpecialCharTok{{-}}\DecValTok{9}\NormalTok{, }\DecValTok{9}\NormalTok{, }\DecValTok{1}\NormalTok{, }\AttributeTok{xlab=}\StringTok{"x1"}\NormalTok{)}
\NormalTok{util}\SpecialCharTok{$}\FunctionTok{plot\_conditional\_median\_quantiles}\NormalTok{(samples, pred\_names, data}\SpecialCharTok{$}\NormalTok{X[,}\DecValTok{2}\NormalTok{],}
                                       \SpecialCharTok{{-}}\DecValTok{9}\NormalTok{, }\DecValTok{9}\NormalTok{, }\DecValTok{1}\NormalTok{, }\AttributeTok{xlab=}\StringTok{"x2"}\NormalTok{)}
\NormalTok{util}\SpecialCharTok{$}\FunctionTok{plot\_conditional\_median\_quantiles}\NormalTok{(samples, pred\_names, data}\SpecialCharTok{$}\NormalTok{X[,}\DecValTok{3}\NormalTok{],}
                                       \SpecialCharTok{{-}}\DecValTok{9}\NormalTok{, }\DecValTok{9}\NormalTok{, }\DecValTok{1}\NormalTok{, }\AttributeTok{xlab=}\StringTok{"x3"}\NormalTok{)}
\end{Highlighting}
\end{Shaded}

\includegraphics{demo_files/figure-pdf/unnamed-chunk-27-1.pdf}

\subsection{Posterior Inference}\label{posterior-inference-1}

Now we're ready to incorporate the observed data.

\begin{codelisting}

\caption{\texttt{multi\textbackslash\_full\textbackslash\_model.stan}}

\begin{Shaded}
\begin{Highlighting}[]
\KeywordTok{data}\NormalTok{ \{}
  \DataTypeTok{int}\NormalTok{\textless{}}\KeywordTok{lower}\NormalTok{=}\DecValTok{0}\NormalTok{\textgreater{} M;  }\CommentTok{// Number of covariates}
  \DataTypeTok{int}\NormalTok{\textless{}}\KeywordTok{lower}\NormalTok{=}\DecValTok{0}\NormalTok{\textgreater{} N;  }\CommentTok{// Number of observations}
  
  \DataTypeTok{vector}\NormalTok{[M] x0;    }\CommentTok{// Covariate baselines}
  \DataTypeTok{matrix}\NormalTok{[N, M] X;  }\CommentTok{// Covariate design matrix}
  
  \DataTypeTok{array}\NormalTok{[N] }\DataTypeTok{real}\NormalTok{ y; }\CommentTok{// Variates}
\NormalTok{\}}

\KeywordTok{transformed data}\NormalTok{ \{}
  \DataTypeTok{matrix}\NormalTok{[N, M * (M + }\DecValTok{3}\NormalTok{) / }\DecValTok{2}\NormalTok{ + }\DecValTok{1}\NormalTok{] deltaX;}
  \ControlFlowTok{for}\NormalTok{ (n }\ControlFlowTok{in} \DecValTok{1}\NormalTok{:N) \{}
\NormalTok{    deltaX[n, }\DecValTok{1}\NormalTok{] = }\DecValTok{1}\NormalTok{;}
    
    \ControlFlowTok{for}\NormalTok{ (m1 }\ControlFlowTok{in} \DecValTok{1}\NormalTok{:M) \{}
      \CommentTok{// Linear perturbations}
\NormalTok{      deltaX[n, m1 + }\DecValTok{1}\NormalTok{] = X[n, m1] {-} x0[m1];}
\NormalTok{    \}}
    
    \ControlFlowTok{for}\NormalTok{ (m1 }\ControlFlowTok{in} \DecValTok{1}\NormalTok{:M) \{}
      \CommentTok{// On{-}diagonal quadratic perturbations}
\NormalTok{      deltaX[n, M + m1 + }\DecValTok{1}\NormalTok{] }
\NormalTok{        = deltaX[n, m1 + }\DecValTok{1}\NormalTok{] * deltaX[n, m1 + }\DecValTok{1}\NormalTok{];}
  
      \ControlFlowTok{for}\NormalTok{ (m2 }\ControlFlowTok{in}\NormalTok{ (m1 + }\DecValTok{1}\NormalTok{):M) \{}
        \DataTypeTok{int}\NormalTok{ m3 = (}\DecValTok{2}\NormalTok{ * M {-} m1) * (m1 {-} }\DecValTok{1}\NormalTok{) / }\DecValTok{2}\NormalTok{ + m2 {-} m1;}
          
        \CommentTok{// Off{-}diagonal quadratic perturbations}
        \CommentTok{// Factor of 2 ensures that beta parameters have the}
        \CommentTok{// same interpretation as the expanded implementation}
\NormalTok{        deltaX[n, }\DecValTok{2}\NormalTok{ * M + m3 + }\DecValTok{1}\NormalTok{] }
\NormalTok{          = }\DecValTok{2}\NormalTok{ * deltaX[n, m1 + }\DecValTok{1}\NormalTok{] * deltaX[n, m2 + }\DecValTok{1}\NormalTok{];}
\NormalTok{      \}}
\NormalTok{    \}}
\NormalTok{  \}}
\NormalTok{\}}

\KeywordTok{parameters}\NormalTok{ \{}
  \DataTypeTok{real}\NormalTok{ beta0;                      }\CommentTok{// Intercept}
  \DataTypeTok{vector}\NormalTok{[M] beta1;                 }\CommentTok{// Linear slopes}
  \DataTypeTok{vector}\NormalTok{[M] beta2\_d;               }\CommentTok{// On{-}diagonal quadratic slopes}
  \DataTypeTok{vector}\NormalTok{[M * (M {-} }\DecValTok{1}\NormalTok{) / }\DecValTok{2}\NormalTok{] beta2\_o; }\CommentTok{// Off{-}diagonal quadratic slopes}
  \DataTypeTok{real}\NormalTok{\textless{}}\KeywordTok{lower}\NormalTok{=}\DecValTok{0}\NormalTok{\textgreater{} sigma;             }\CommentTok{// Measurement Variability}
\NormalTok{\}}

\KeywordTok{model}\NormalTok{ \{}
  \DataTypeTok{vector}\NormalTok{[M * (M + }\DecValTok{3}\NormalTok{) / }\DecValTok{2}\NormalTok{ + }\DecValTok{1}\NormalTok{] beta}
\NormalTok{    = append\_row(}
\NormalTok{        append\_row(}
\NormalTok{          append\_row(beta0, beta1), }
\NormalTok{        beta2\_d), }
\NormalTok{      beta2\_o);}
  
  \CommentTok{// Prior model}
\NormalTok{  beta0 \textasciitilde{} normal(}\DecValTok{0}\NormalTok{, }\DecValTok{10}\NormalTok{);}
\NormalTok{  beta1 \textasciitilde{} normal(}\DecValTok{0}\NormalTok{, }\DecValTok{10}\NormalTok{);}
\NormalTok{  beta2\_d \textasciitilde{} normal(}\DecValTok{0}\NormalTok{, }\DecValTok{2}\NormalTok{);}
\NormalTok{  beta2\_o \textasciitilde{} normal(}\DecValTok{0}\NormalTok{, }\DecValTok{1}\NormalTok{);}
\NormalTok{  sigma \textasciitilde{} normal(}\DecValTok{0}\NormalTok{, }\DecValTok{5}\NormalTok{);}

  \CommentTok{// Observation model}
\NormalTok{  y \textasciitilde{} normal(deltaX * beta, sigma);}
\NormalTok{\}}

\KeywordTok{generated quantities}\NormalTok{ \{}
  \CommentTok{// Posterior predictions}
  \DataTypeTok{array}\NormalTok{[N] }\DataTypeTok{real}\NormalTok{ y\_pred;}
\NormalTok{  \{}
    \DataTypeTok{vector}\NormalTok{[M * (M + }\DecValTok{3}\NormalTok{) / }\DecValTok{2}\NormalTok{ + }\DecValTok{1}\NormalTok{] beta}
\NormalTok{      = append\_row(}
\NormalTok{          append\_row(}
\NormalTok{            append\_row(beta0, beta1), }
\NormalTok{          beta2\_d),}
\NormalTok{        beta2\_o);}
\NormalTok{    y\_pred = normal\_rng(deltaX * beta, sigma);}
\NormalTok{  \}}
\NormalTok{\}}
\end{Highlighting}
\end{Shaded}

\end{codelisting}

\begin{Shaded}
\begin{Highlighting}[]
\NormalTok{fit }\OtherTok{\textless{}{-}} \FunctionTok{stan}\NormalTok{(}\AttributeTok{file=}\StringTok{\textquotesingle{}stan\_programs/multi\_full\_model.stan\textquotesingle{}}\NormalTok{,}
            \AttributeTok{data=}\NormalTok{data, }\AttributeTok{seed=}\DecValTok{5838299}\NormalTok{,}
            \AttributeTok{warmup=}\DecValTok{1000}\NormalTok{, }\AttributeTok{iter=}\DecValTok{2024}\NormalTok{, }\AttributeTok{refresh=}\DecValTok{0}\NormalTok{)}
\end{Highlighting}
\end{Shaded}

Fortunately our computational fortune has persisted.

\begin{Shaded}
\begin{Highlighting}[]
\NormalTok{diagnostics }\OtherTok{\textless{}{-}}\NormalTok{ util}\SpecialCharTok{$}\FunctionTok{extract\_hmc\_diagnostics}\NormalTok{(fit)}
\NormalTok{util}\SpecialCharTok{$}\FunctionTok{check\_all\_hmc\_diagnostics}\NormalTok{(diagnostics)}
\end{Highlighting}
\end{Shaded}

\begin{verbatim}
  All Hamiltonian Monte Carlo diagnostics are consistent with reliable
Markov chain Monte Carlo.
\end{verbatim}

\begin{Shaded}
\begin{Highlighting}[]
\NormalTok{samples }\OtherTok{\textless{}{-}}\NormalTok{ util}\SpecialCharTok{$}\FunctionTok{extract\_expectand\_vals}\NormalTok{(fit)}
\NormalTok{base\_samples }\OtherTok{\textless{}{-}}\NormalTok{ util}\SpecialCharTok{$}\FunctionTok{filter\_expectands}\NormalTok{(samples,}
                                       \FunctionTok{c}\NormalTok{(}\StringTok{\textquotesingle{}beta0\textquotesingle{}}\NormalTok{, }\StringTok{\textquotesingle{}beta1\textquotesingle{}}\NormalTok{,}
                                         \StringTok{\textquotesingle{}beta2\_d\textquotesingle{}}\NormalTok{, }\StringTok{\textquotesingle{}beta2\_o\textquotesingle{}}\NormalTok{,}
                                         \StringTok{\textquotesingle{}sigma\textquotesingle{}}\NormalTok{),}
                                       \ConstantTok{TRUE}\NormalTok{)}
\NormalTok{util}\SpecialCharTok{$}\FunctionTok{check\_all\_expectand\_diagnostics}\NormalTok{(base\_samples)}
\end{Highlighting}
\end{Shaded}

\begin{verbatim}
All expectands checked appear to be behaving well enough for reliable
Markov chain Monte Carlo estimation.
\end{verbatim}

The summary statistics that we used above to implement our prior
predictive checks are equally useful for implementing informative
posterior retrodictive checks. Conveniently the visualization functions
all feature \texttt{baseline\_values} functions that we can use to
visualize the observed behavior along with the posterior predictive
behavior.

\begin{Shaded}
\begin{Highlighting}[]
\FunctionTok{par}\NormalTok{(}\AttributeTok{mfrow=}\FunctionTok{c}\NormalTok{(}\DecValTok{1}\NormalTok{, }\DecValTok{1}\NormalTok{))}

\NormalTok{util}\SpecialCharTok{$}\FunctionTok{plot\_hist\_quantiles}\NormalTok{(samples, }\StringTok{\textquotesingle{}y\_pred\textquotesingle{}}\NormalTok{, }\AttributeTok{baseline\_values=}\NormalTok{data}\SpecialCharTok{$}\NormalTok{y)}
\end{Highlighting}
\end{Shaded}

\includegraphics{demo_files/figure-pdf/unnamed-chunk-30-1.pdf}

Additionally the \texttt{plot\_conditional\_mean\_quantiles} and
\texttt{plot\_conditional\_median\_quantiles} functions feature a
\texttt{residual} option that plots the posterior predictive behaviors
relative to the baseline values. Any deviations from zero in these plots
suggests retrodictive tension; here, however, there don't seem to be any
problems.

\begin{Shaded}
\begin{Highlighting}[]
\FunctionTok{par}\NormalTok{(}\AttributeTok{mfrow=}\FunctionTok{c}\NormalTok{(}\DecValTok{2}\NormalTok{, }\DecValTok{3}\NormalTok{))}

\NormalTok{pred\_names }\OtherTok{\textless{}{-}}\NormalTok{ util}\SpecialCharTok{$}\FunctionTok{name\_array}\NormalTok{(}\StringTok{\textquotesingle{}y\_pred\textquotesingle{}}\NormalTok{, }\FunctionTok{c}\NormalTok{(data}\SpecialCharTok{$}\NormalTok{N))}
\NormalTok{util}\SpecialCharTok{$}\FunctionTok{plot\_conditional\_mean\_quantiles}\NormalTok{(samples, pred\_names, data}\SpecialCharTok{$}\NormalTok{X[,}\DecValTok{1}\NormalTok{],}
                                     \SpecialCharTok{{-}}\DecValTok{9}\NormalTok{, }\DecValTok{9}\NormalTok{, }\DecValTok{1}\NormalTok{, data}\SpecialCharTok{$}\NormalTok{y, }\AttributeTok{xlab=}\StringTok{"x1"}\NormalTok{)}
\NormalTok{util}\SpecialCharTok{$}\FunctionTok{plot\_conditional\_mean\_quantiles}\NormalTok{(samples, pred\_names, data}\SpecialCharTok{$}\NormalTok{X[,}\DecValTok{2}\NormalTok{],}
                                     \SpecialCharTok{{-}}\DecValTok{9}\NormalTok{, }\DecValTok{9}\NormalTok{, }\DecValTok{1}\NormalTok{, data}\SpecialCharTok{$}\NormalTok{y, }\AttributeTok{xlab=}\StringTok{"x2"}\NormalTok{)}
\NormalTok{util}\SpecialCharTok{$}\FunctionTok{plot\_conditional\_mean\_quantiles}\NormalTok{(samples, pred\_names, data}\SpecialCharTok{$}\NormalTok{X[,}\DecValTok{3}\NormalTok{],}
                                     \SpecialCharTok{{-}}\DecValTok{9}\NormalTok{, }\DecValTok{9}\NormalTok{, }\DecValTok{1}\NormalTok{, data}\SpecialCharTok{$}\NormalTok{y, }\AttributeTok{xlab=}\StringTok{"x3"}\NormalTok{)}

\NormalTok{util}\SpecialCharTok{$}\FunctionTok{plot\_conditional\_mean\_quantiles}\NormalTok{(samples, pred\_names, data}\SpecialCharTok{$}\NormalTok{X[,}\DecValTok{1}\NormalTok{],}
                                     \SpecialCharTok{{-}}\DecValTok{9}\NormalTok{, }\DecValTok{9}\NormalTok{, }\DecValTok{1}\NormalTok{, data}\SpecialCharTok{$}\NormalTok{y, }\AttributeTok{residual=}\ConstantTok{TRUE}\NormalTok{,}
                                     \AttributeTok{xlab=}\StringTok{"x1"}\NormalTok{)}
\NormalTok{util}\SpecialCharTok{$}\FunctionTok{plot\_conditional\_mean\_quantiles}\NormalTok{(samples, pred\_names, data}\SpecialCharTok{$}\NormalTok{X[,}\DecValTok{2}\NormalTok{],}
                                     \SpecialCharTok{{-}}\DecValTok{9}\NormalTok{, }\DecValTok{9}\NormalTok{, }\DecValTok{1}\NormalTok{, data}\SpecialCharTok{$}\NormalTok{y, }\AttributeTok{residual=}\ConstantTok{TRUE}\NormalTok{,}
                                     \AttributeTok{xlab=}\StringTok{"x2"}\NormalTok{)}
\NormalTok{util}\SpecialCharTok{$}\FunctionTok{plot\_conditional\_mean\_quantiles}\NormalTok{(samples, pred\_names, data}\SpecialCharTok{$}\NormalTok{X[,}\DecValTok{3}\NormalTok{],}
                                     \SpecialCharTok{{-}}\DecValTok{9}\NormalTok{, }\DecValTok{9}\NormalTok{, }\DecValTok{1}\NormalTok{, data}\SpecialCharTok{$}\NormalTok{y, }\AttributeTok{residual=}\ConstantTok{TRUE}\NormalTok{,}
                                     \AttributeTok{xlab=}\StringTok{"x3"}\NormalTok{)}
\end{Highlighting}
\end{Shaded}

\includegraphics{demo_files/figure-pdf/unnamed-chunk-31-1.pdf}

\begin{Shaded}
\begin{Highlighting}[]
\FunctionTok{par}\NormalTok{(}\AttributeTok{mfrow=}\FunctionTok{c}\NormalTok{(}\DecValTok{2}\NormalTok{, }\DecValTok{3}\NormalTok{))}

\NormalTok{pred\_names }\OtherTok{\textless{}{-}}\NormalTok{ util}\SpecialCharTok{$}\FunctionTok{name\_array}\NormalTok{(}\StringTok{\textquotesingle{}y\_pred\textquotesingle{}}\NormalTok{, }\FunctionTok{c}\NormalTok{(data}\SpecialCharTok{$}\NormalTok{N))}
\NormalTok{util}\SpecialCharTok{$}\FunctionTok{plot\_conditional\_median\_quantiles}\NormalTok{(samples, pred\_names, data}\SpecialCharTok{$}\NormalTok{X[,}\DecValTok{1}\NormalTok{],}
                                       \SpecialCharTok{{-}}\DecValTok{9}\NormalTok{, }\DecValTok{9}\NormalTok{, }\DecValTok{1}\NormalTok{, data}\SpecialCharTok{$}\NormalTok{y, }\AttributeTok{xlab=}\StringTok{"x1"}\NormalTok{)}
\NormalTok{util}\SpecialCharTok{$}\FunctionTok{plot\_conditional\_median\_quantiles}\NormalTok{(samples, pred\_names, data}\SpecialCharTok{$}\NormalTok{X[,}\DecValTok{2}\NormalTok{],}
                                       \SpecialCharTok{{-}}\DecValTok{9}\NormalTok{, }\DecValTok{9}\NormalTok{, }\DecValTok{1}\NormalTok{, data}\SpecialCharTok{$}\NormalTok{y, }\AttributeTok{xlab=}\StringTok{"x2"}\NormalTok{)}
\NormalTok{util}\SpecialCharTok{$}\FunctionTok{plot\_conditional\_median\_quantiles}\NormalTok{(samples, pred\_names, data}\SpecialCharTok{$}\NormalTok{X[,}\DecValTok{3}\NormalTok{],}
                                       \SpecialCharTok{{-}}\DecValTok{9}\NormalTok{, }\DecValTok{9}\NormalTok{, }\DecValTok{1}\NormalTok{, data}\SpecialCharTok{$}\NormalTok{y, }\AttributeTok{xlab=}\StringTok{"x3"}\NormalTok{)}

\NormalTok{util}\SpecialCharTok{$}\FunctionTok{plot\_conditional\_median\_quantiles}\NormalTok{(samples, pred\_names, data}\SpecialCharTok{$}\NormalTok{X[,}\DecValTok{1}\NormalTok{],}
                                       \SpecialCharTok{{-}}\DecValTok{9}\NormalTok{, }\DecValTok{9}\NormalTok{, }\DecValTok{1}\NormalTok{, data}\SpecialCharTok{$}\NormalTok{y, }\AttributeTok{residual=}\ConstantTok{TRUE}\NormalTok{,}
                                       \AttributeTok{xlab=}\StringTok{"x1"}\NormalTok{)}
\NormalTok{util}\SpecialCharTok{$}\FunctionTok{plot\_conditional\_median\_quantiles}\NormalTok{(samples, pred\_names, data}\SpecialCharTok{$}\NormalTok{X[,}\DecValTok{2}\NormalTok{],}
                                       \SpecialCharTok{{-}}\DecValTok{9}\NormalTok{, }\DecValTok{9}\NormalTok{, }\DecValTok{1}\NormalTok{, data}\SpecialCharTok{$}\NormalTok{y, }\AttributeTok{residual=}\ConstantTok{TRUE}\NormalTok{,}
                                       \AttributeTok{xlab=}\StringTok{"x2"}\NormalTok{)}
\NormalTok{util}\SpecialCharTok{$}\FunctionTok{plot\_conditional\_median\_quantiles}\NormalTok{(samples, pred\_names, data}\SpecialCharTok{$}\NormalTok{X[,}\DecValTok{3}\NormalTok{],}
                                       \SpecialCharTok{{-}}\DecValTok{9}\NormalTok{, }\DecValTok{9}\NormalTok{, }\DecValTok{1}\NormalTok{, data}\SpecialCharTok{$}\NormalTok{y, }\AttributeTok{residual=}\ConstantTok{TRUE}\NormalTok{,}
                                       \AttributeTok{xlab=}\StringTok{"x3"}\NormalTok{)}
\end{Highlighting}
\end{Shaded}

\includegraphics{demo_files/figure-pdf/unnamed-chunk-32-1.pdf}

With no indications of model inadequacy we can move onto our posterior
inferences. As before we can visualize the pushforward posterior
distributions for each individual, one-dimensional parameter.

\begin{Shaded}
\begin{Highlighting}[]
\FunctionTok{par}\NormalTok{(}\AttributeTok{mfrow=}\FunctionTok{c}\NormalTok{(}\DecValTok{4}\NormalTok{, }\DecValTok{3}\NormalTok{))}

\NormalTok{util}\SpecialCharTok{$}\FunctionTok{plot\_expectand\_pushforward}\NormalTok{(samples[[}\StringTok{"beta0"}\NormalTok{]], }\DecValTok{25}\NormalTok{,}
                                \AttributeTok{display\_name=}\StringTok{"beta0"}\NormalTok{)}

\FunctionTok{plot.new}\NormalTok{()}

\NormalTok{util}\SpecialCharTok{$}\FunctionTok{plot\_expectand\_pushforward}\NormalTok{(samples[[}\StringTok{"sigma"}\NormalTok{]], }\DecValTok{25}\NormalTok{,}
                                \AttributeTok{display\_name=}\StringTok{"sigma"}\NormalTok{)}

\ControlFlowTok{for}\NormalTok{ (m }\ControlFlowTok{in} \DecValTok{1}\SpecialCharTok{:}\NormalTok{data}\SpecialCharTok{$}\NormalTok{M) \{}
\NormalTok{  name }\OtherTok{\textless{}{-}} \FunctionTok{paste0}\NormalTok{(}\StringTok{\textquotesingle{}beta1[\textquotesingle{}}\NormalTok{, m, }\StringTok{\textquotesingle{}]\textquotesingle{}}\NormalTok{)}
\NormalTok{  util}\SpecialCharTok{$}\FunctionTok{plot\_expectand\_pushforward}\NormalTok{(samples[[name]], }\DecValTok{25}\NormalTok{,}
                                  \AttributeTok{display\_name=}\NormalTok{name)}
\NormalTok{\}}

\ControlFlowTok{for}\NormalTok{ (m }\ControlFlowTok{in} \DecValTok{1}\SpecialCharTok{:}\NormalTok{data}\SpecialCharTok{$}\NormalTok{M) \{}
\NormalTok{  name }\OtherTok{\textless{}{-}} \FunctionTok{paste0}\NormalTok{(}\StringTok{\textquotesingle{}beta2\_d[\textquotesingle{}}\NormalTok{, m, }\StringTok{\textquotesingle{}]\textquotesingle{}}\NormalTok{)}
\NormalTok{  util}\SpecialCharTok{$}\FunctionTok{plot\_expectand\_pushforward}\NormalTok{(samples[[name]], }\DecValTok{25}\NormalTok{,}
                                  \AttributeTok{display\_name=}\NormalTok{name)}
\NormalTok{\}}

\ControlFlowTok{for}\NormalTok{ (m }\ControlFlowTok{in} \DecValTok{1}\SpecialCharTok{:}\NormalTok{data}\SpecialCharTok{$}\NormalTok{M) \{}
\NormalTok{  name }\OtherTok{\textless{}{-}} \FunctionTok{paste0}\NormalTok{(}\StringTok{\textquotesingle{}beta2\_o[\textquotesingle{}}\NormalTok{, m, }\StringTok{\textquotesingle{}]\textquotesingle{}}\NormalTok{)}
\NormalTok{  util}\SpecialCharTok{$}\FunctionTok{plot\_expectand\_pushforward}\NormalTok{(samples[[name]], }\DecValTok{25}\NormalTok{,}
                                  \AttributeTok{display\_name=}\NormalTok{name)}
\NormalTok{\}}
\end{Highlighting}
\end{Shaded}

\includegraphics{demo_files/figure-pdf/unnamed-chunk-33-1.pdf}

The \texttt{plot\_disc\_pushforward\_quantiles} function plots
disconnected, marginal nested quantile intervals for a collection of
one-dimensional variables. This allows for a more compact visualization
of the marginal posterior distributions.

\begin{Shaded}
\begin{Highlighting}[]
\FunctionTok{par}\NormalTok{(}\AttributeTok{mfrow=}\FunctionTok{c}\NormalTok{(}\DecValTok{1}\NormalTok{, }\DecValTok{3}\NormalTok{))}

\NormalTok{names }\OtherTok{\textless{}{-}} \FunctionTok{sapply}\NormalTok{(}\DecValTok{1}\SpecialCharTok{:}\NormalTok{data}\SpecialCharTok{$}\NormalTok{M, }\ControlFlowTok{function}\NormalTok{(m) }\FunctionTok{paste0}\NormalTok{(}\StringTok{\textquotesingle{}beta1[\textquotesingle{}}\NormalTok{, m, }\StringTok{\textquotesingle{}]\textquotesingle{}}\NormalTok{))}
\NormalTok{util}\SpecialCharTok{$}\FunctionTok{plot\_disc\_pushforward\_quantiles}\NormalTok{(samples, names,}
                                     \AttributeTok{xlab=}\StringTok{"beta1"}\NormalTok{,}
                                     \AttributeTok{ylab=}\StringTok{"Marginal Posterior Quantiles"}\NormalTok{)}

\NormalTok{names }\OtherTok{\textless{}{-}} \FunctionTok{sapply}\NormalTok{(}\DecValTok{1}\SpecialCharTok{:}\NormalTok{data}\SpecialCharTok{$}\NormalTok{M, }\ControlFlowTok{function}\NormalTok{(m) }\FunctionTok{paste0}\NormalTok{(}\StringTok{\textquotesingle{}beta2\_d[\textquotesingle{}}\NormalTok{, m, }\StringTok{\textquotesingle{}]\textquotesingle{}}\NormalTok{))}
\NormalTok{util}\SpecialCharTok{$}\FunctionTok{plot\_disc\_pushforward\_quantiles}\NormalTok{(samples, names,}
                                     \AttributeTok{xlab=}\StringTok{"beta2\_d"}\NormalTok{,}
                                     \AttributeTok{ylab=}\StringTok{"Marginal Posterior Quantiles"}\NormalTok{)}

\NormalTok{names }\OtherTok{\textless{}{-}} \FunctionTok{sapply}\NormalTok{(}\DecValTok{1}\SpecialCharTok{:}\NormalTok{data}\SpecialCharTok{$}\NormalTok{M, }\ControlFlowTok{function}\NormalTok{(m) }\FunctionTok{paste0}\NormalTok{(}\StringTok{\textquotesingle{}beta2\_o[\textquotesingle{}}\NormalTok{, m, }\StringTok{\textquotesingle{}]\textquotesingle{}}\NormalTok{))}
\NormalTok{util}\SpecialCharTok{$}\FunctionTok{plot\_disc\_pushforward\_quantiles}\NormalTok{(samples, names,}
                                     \AttributeTok{xlab=}\StringTok{"beta2\_o"}\NormalTok{,}
                                     \AttributeTok{ylab=}\StringTok{"Marginal Posterior Quantiles"}\NormalTok{)}
\end{Highlighting}
\end{Shaded}

\includegraphics{demo_files/figure-pdf/unnamed-chunk-34-1.pdf}

\begin{Shaded}
\begin{Highlighting}[]
\FunctionTok{par}\NormalTok{(}\AttributeTok{mfrow=}\FunctionTok{c}\NormalTok{(}\DecValTok{1}\NormalTok{, }\DecValTok{1}\NormalTok{))}

\NormalTok{names }\OtherTok{\textless{}{-}} \FunctionTok{c}\NormalTok{(}\FunctionTok{sapply}\NormalTok{(}\DecValTok{1}\SpecialCharTok{:}\NormalTok{data}\SpecialCharTok{$}\NormalTok{M, }\ControlFlowTok{function}\NormalTok{(m) }\FunctionTok{paste0}\NormalTok{(}\StringTok{\textquotesingle{}beta1[\textquotesingle{}}\NormalTok{, m, }\StringTok{\textquotesingle{}]\textquotesingle{}}\NormalTok{)),}
           \FunctionTok{sapply}\NormalTok{(}\DecValTok{1}\SpecialCharTok{:}\NormalTok{data}\SpecialCharTok{$}\NormalTok{M, }\ControlFlowTok{function}\NormalTok{(m) }\FunctionTok{paste0}\NormalTok{(}\StringTok{\textquotesingle{}beta2\_d[\textquotesingle{}}\NormalTok{, m, }\StringTok{\textquotesingle{}]\textquotesingle{}}\NormalTok{)),}
           \FunctionTok{sapply}\NormalTok{(}\DecValTok{1}\SpecialCharTok{:}\NormalTok{data}\SpecialCharTok{$}\NormalTok{M, }\ControlFlowTok{function}\NormalTok{(m) }\FunctionTok{paste0}\NormalTok{(}\StringTok{\textquotesingle{}beta2\_o[\textquotesingle{}}\NormalTok{, m, }\StringTok{\textquotesingle{}]\textquotesingle{}}\NormalTok{)))}
\NormalTok{util}\SpecialCharTok{$}\FunctionTok{plot\_disc\_pushforward\_quantiles}\NormalTok{(samples, names,}
                                     \AttributeTok{xlab=}\StringTok{"All Slopes"}\NormalTok{,}
                                     \AttributeTok{ylab=}\StringTok{"Marginal Posterior Quantiles"}\NormalTok{)}
\end{Highlighting}
\end{Shaded}

\includegraphics{demo_files/figure-pdf/unnamed-chunk-35-1.pdf}

This function also includes an optional \texttt{baseline\_values}
argument and \texttt{residual} configuration which we can use to compare
the probabilistic to the point values, for example our marginal
posterior inferences to the true values when analyzing simulated data.

\begin{Shaded}
\begin{Highlighting}[]
\NormalTok{true\_slopes }\OtherTok{\textless{}{-}} \FunctionTok{c}\NormalTok{(}\SpecialCharTok{{-}}\FloatTok{6.00}\NormalTok{, }\SpecialCharTok{{-}}\FloatTok{1.50}\NormalTok{, }\FloatTok{13.00}\NormalTok{,  }\FloatTok{0.50}\NormalTok{, }\FloatTok{0.25}\NormalTok{,}
                 \FloatTok{1.00}\NormalTok{, }\SpecialCharTok{{-}}\FloatTok{0.50}\NormalTok{, }\SpecialCharTok{{-}}\FloatTok{2.00}\NormalTok{, }\SpecialCharTok{{-}}\FloatTok{1.00}\NormalTok{)}

\FunctionTok{par}\NormalTok{(}\AttributeTok{mfrow=}\FunctionTok{c}\NormalTok{(}\DecValTok{2}\NormalTok{, }\DecValTok{1}\NormalTok{))}

\NormalTok{util}\SpecialCharTok{$}\FunctionTok{plot\_disc\_pushforward\_quantiles}\NormalTok{(samples, names,}
                                     \AttributeTok{baseline\_values=}\NormalTok{true\_slopes,}
                                     \AttributeTok{baseline\_col=}\NormalTok{util}\SpecialCharTok{$}\NormalTok{c\_mid\_teal,}
                                     \AttributeTok{xlab=}\StringTok{"All Slopes"}\NormalTok{,}
                                     \AttributeTok{ylab=}\StringTok{"Marginal Posterior}\SpecialCharTok{\textbackslash{}n}\StringTok{Quantiles"}\NormalTok{)}

\NormalTok{util}\SpecialCharTok{$}\FunctionTok{plot\_disc\_pushforward\_quantiles}\NormalTok{(samples, names,}
                                     \AttributeTok{baseline\_values=}\NormalTok{true\_slopes,}
                                     \AttributeTok{residual=}\ConstantTok{TRUE}\NormalTok{,}
                                     \AttributeTok{xlab=}\StringTok{"All Slopes"}\NormalTok{,}
                                     \AttributeTok{ylab=}\StringTok{"Marginal Posterior}\SpecialCharTok{\textbackslash{}n}\StringTok{Quantile Residuals"}\NormalTok{)}
\end{Highlighting}
\end{Shaded}

\includegraphics{demo_files/figure-pdf/unnamed-chunk-36-1.pdf}

\section*{License}\label{license}
\addcontentsline{toc}{section}{License}

The code in this case study is copyrighted by Michael Betancourt and
licensed under the new BSD (3-clause) license:

https://opensource.org/licenses/BSD-3-Clause

The text and figures in this case study are copyrighted by Michael
Betancourt and licensed under the CC BY-NC 4.0 license:

https://creativecommons.org/licenses/by-nc/4.0/

\section*{Original Computing
Environment}\label{original-computing-environment}
\addcontentsline{toc}{section}{Original Computing Environment}

\begin{Shaded}
\begin{Highlighting}[]
\FunctionTok{writeLines}\NormalTok{(}\FunctionTok{readLines}\NormalTok{(}\FunctionTok{file.path}\NormalTok{(}\FunctionTok{Sys.getenv}\NormalTok{(}\StringTok{"HOME"}\NormalTok{), }\StringTok{".R/Makevars"}\NormalTok{)))}
\end{Highlighting}
\end{Shaded}

\begin{verbatim}
CC=clang

CXXFLAGS=-O3 -mtune=native -march=native -Wno-unused-variable -Wno-unused-function -Wno-macro-redefined -Wno-unneeded-internal-declaration
CXX=clang++ -arch x86_64 -ftemplate-depth-256

CXX14FLAGS=-O3 -mtune=native -march=native -Wno-unused-variable -Wno-unused-function -Wno-macro-redefined -Wno-unneeded-internal-declaration -Wno-unknown-pragmas
CXX14=clang++ -arch x86_64 -ftemplate-depth-256
\end{verbatim}

\begin{Shaded}
\begin{Highlighting}[]
\FunctionTok{sessionInfo}\NormalTok{()}
\end{Highlighting}
\end{Shaded}

\begin{verbatim}
R version 4.3.2 (2023-10-31)
Platform: x86_64-apple-darwin20 (64-bit)
Running under: macOS Sonoma 14.4.1

Matrix products: default
BLAS:   /Library/Frameworks/R.framework/Versions/4.3-x86_64/Resources/lib/libRblas.0.dylib 
LAPACK: /Library/Frameworks/R.framework/Versions/4.3-x86_64/Resources/lib/libRlapack.dylib;  LAPACK version 3.11.0

locale:
[1] en_US.UTF-8/en_US.UTF-8/en_US.UTF-8/C/en_US.UTF-8/en_US.UTF-8

time zone: America/New_York
tzcode source: internal

attached base packages:
[1] stats     graphics  grDevices utils     datasets  methods   base     

other attached packages:
[1] colormap_0.1.4     rstan_2.32.6       StanHeaders_2.32.7

loaded via a namespace (and not attached):
 [1] gtable_0.3.4       jsonlite_1.8.8     compiler_4.3.2     Rcpp_1.0.11       
 [5] stringr_1.5.1      parallel_4.3.2     gridExtra_2.3      scales_1.3.0      
 [9] yaml_2.3.8         fastmap_1.1.1      ggplot2_3.4.4      R6_2.5.1          
[13] curl_5.2.0         knitr_1.45         tibble_3.2.1       munsell_0.5.0     
[17] pillar_1.9.0       rlang_1.1.2        utf8_1.2.4         V8_4.4.1          
[21] stringi_1.8.3      inline_0.3.19      xfun_0.41          RcppParallel_5.1.7
[25] cli_3.6.2          magrittr_2.0.3     digest_0.6.33      grid_4.3.2        
[29] lifecycle_1.0.4    vctrs_0.6.5        evaluate_0.23      glue_1.6.2        
[33] QuickJSR_1.0.8     codetools_0.2-19   stats4_4.3.2       pkgbuild_1.4.3    
[37] fansi_1.0.6        colorspace_2.1-0   rmarkdown_2.25     matrixStats_1.2.0 
[41] tools_4.3.2        loo_2.6.0          pkgconfig_2.0.3    htmltools_0.5.7   
\end{verbatim}



\end{document}
